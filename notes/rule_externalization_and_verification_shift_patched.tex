\documentclass[11pt]{article}


\usepackage[T1]{fontenc}
\usepackage[utf8]{inputenc} % for pdfLaTeX
\usepackage{lmodern}        % vector fonts; fixes copy/paste artifacts
\usepackage{microtype}      % better spacing and fewer overfull boxes

\usepackage{amsmath, amssymb, amsthm}
\usepackage{geometry}
\usepackage{hyperref}
\hypersetup{
  pdftitle={Rule Externalization and the Decision--Verification Shift},
  pdfauthor={Yanlin Li},
  colorlinks=true,
  linkcolor=blue,
  urlcolor=blue,
  citecolor=blue
}

\geometry{margin=1in}

\title{Rule Externalization and the Decision--Verification Shift}
\author{YANLIN LI\\
\small Independent Researcher}
\date{}

\theoremstyle{definition}
\newtheorem{definition}{Definition}
\newtheorem{observation}{Observation}
\newtheorem{proposition}{Proposition}

\begin{document}
\maketitle

\begin{abstract}
We examine the structural role of rule availability in computational decision
problems. By modeling rules as explicit parameters, we formalize a distinction
between direct decision under a fixed and internalized rule structure and
verification when such structure is externalized or left implicit. We show that
externalization of rules induces a shift from direct decision procedures to
verification relative to auxiliary information, yielding a problem formulation
characteristic of nondeterministic polynomial-time verification. No claims are
made regarding separations between complexity classes; the analysis is purely
structural.
\end{abstract}

\section{Introduction}

Decision problems in complexity theory are typically formulated relative to a
fixed and shared rule structure. Under this assumption, the computational task
is to decide membership of an instance in a language using a procedure that
implicitly incorporates those rules.

\noindent
In practice and in theory, however, rules governing validity or interpretation
may be incomplete, implicit, or external to the decision procedure itself. In
such cases, correctness is often established not by direct decision but by
verification relative to additional information encoding the missing structure.

\noindent
This paper formalizes this distinction. We show that when rule structure is
externalized, decision tasks naturally assume a verification-based form.
The resulting formulation aligns with the standard definition of
$\mathbf{NP}$, not as a complexity separation but as a change of interface.

\section{Rule-Parameterized Decision Problems}

Let $\Sigma$ be a finite alphabet.

\begin{definition}[Rule Structure]
A \emph{rule structure} is represented by a finite binary string
$r \in \{0,1\}^*$.
\end{definition}

For each fixed rule structure $r$, let
\[
L_r \subseteq \Sigma^*
\]
denote a language determined by $r$.

The associated decision task is:
\[
\textsf{DECIDE}_r(x) =
\begin{cases}
1 & \text{if } x \in L_r, \\
0 & \text{otherwise}.
\end{cases}
\]

\begin{definition}[Internalized Decision]
The rule structure $r$ is said to be \emph{internalized} if it is fixed and
shared by the decision procedure.

\noindent
The task $\textsf{DECIDE}_r$ admits \emph{direct efficient decision} if there
exists a deterministic polynomial-time algorithm $A_r$ such that
\[
A_r(x) = \textsf{DECIDE}_r(x)
\quad \text{for all } x \in \Sigma^*.
\]
\end{definition}

\noindent
In this setting, the decision procedure operates entirely within the given rule
framework.

\section{Externalized Rules and Verification Interfaces}

We now consider the case in which the rule structure $r$ is not internalized by
the decision procedure.

\noindent
Instead of direct decision, correctness relative to $r$ is established through
auxiliary information.

\begin{definition}[Verification Predicate]
A \emph{verification predicate} is a polynomial-time computable function
\[
V : \{0,1\}^* \times \Sigma^* \times \{0,1\}^* \to \{0,1\},
\]
where $V(r,x,w)=1$ certifies that $x \in L_r$ relative to auxiliary information
$w$.
\end{definition}

\noindent
Using $V$, define the verification language
\[
L^{\exists} =
\{ (r,x) \mid \exists w \in \{0,1\}^* \text{ such that } V(r,x,w)=1 \}.
\]

\begin{proposition}
The language $L^{\exists}$ belongs to $\mathbf{NP}$.
\end{proposition}

\begin{proof}
Membership of $(r,x)$ in $L^{\exists}$ is witnessed by a certificate $w$ whose
validity can be verified in polynomial time by $V$. This satisfies the defining
property of $\mathbf{NP}$.
\end{proof}

\section{Decision--Verification Interface Shift}

The preceding constructions formalize a structural distinction:

\begin{observation}[Rule Externalization and Verification Shift]
Efficient decision procedures presuppose a fixed and explicit rule structure
internal to the problem formulation. When such structure is externalized or left
implicit, the task no longer admits direct decision and instead becomes
verification relative to auxiliary information encoding the missing rules.
\end{observation}

\noindent
This shift reflects a change in computational \emph{interface} rather than in
computational power. The underlying language family remains unchanged, but the
mode by which correctness is established transitions from direct decision to
verification.

\section{Scope and Non-Claims}

This work makes no claims regarding separations between complexity classes,
including $\mathbf{P}$ and $\mathbf{NP}$. The constructions presented here do not
modify standard definitions and do not assert the existence or non-existence of
any class inclusions beyond those definitional in nature.

\noindent
The purpose of this paper is solely to isolate and formalize the structural
effect of rule externalization on decision problem formulation.

\section{Conclusion}

By treating rule structures as explicit parameters, we have shown that the
availability of internalized rules is a necessary condition for direct decision
interfaces. When such rules are externalized, decision problems naturally
assume a verification-based form consistent with nondeterministic polynomial-time
verification.

\noindent
This observation clarifies a structural mechanism underlying the distinction
between decision and verification and may inform further boundary-based analyses
of computational problem formulations.

% No bibliography in this short note.
\end{document}
\section*{Citation}

Li, Yanlin.
\emph{<TITLE>}.
Version~v0.4, 2025.
Available at: \url{https://github.com/ogmf/Type-Consistent-Feasibility/tree/v0.4}.

\noindent Please cite the version you consulted.

\bibliographystyle{plain}
\bibliography{references}
