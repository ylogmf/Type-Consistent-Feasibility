\documentclass[11pt]{article}

\usepackage{amsmath,amssymb,amsthm}
\usepackage{geometry}
\usepackage{enumitem}
\usepackage{hyperref}

\geometry{margin=1in}

% ---------- Environments ----------
\theoremstyle{plain}
\newtheorem{theorem}{Theorem}
\newtheorem{proposition}[theorem]{Proposition}
\newtheorem{corollary}[theorem]{Corollary}
\newtheorem{conjecture}[theorem]{Conjecture}

\theoremstyle{definition}
\newtheorem{definition}{Definition}

\theoremstyle{remark}
\newtheorem{observation}{Observation}
\newtheorem{remark}{Remark}

\title{Rule-Internal Verification and Interface Non-Invertibility}
\author{ }
\date{}


\newcommand{\negl}{\operatorname{negl}}

\begin{document}
\maketitle

\begin{abstract}
This note summarizes a unifying structural perspective that arises across
complexity theory, cryptography, proof systems, and quantum theory.
The central distinction is between \emph{rule-internal} verification interfaces
and \emph{rule-external} generative structure.
We show that many celebrated hardness and no-go results can be understood as
instances of a single phenomenon: interface non-invertibility caused by
dimensional or informational compression at the verification interface.
The contribution is not a new domain-specific theorem, but a formal framework
that makes this shared obstruction explicit.
\end{abstract}

% ============================================================
\section{Rule-Based Verification Framework}

We fix a public rule system $\Theta$.
A verification predicate
\[
V_\Theta(x,w)\in\{0,1\}
\]
runs in deterministic polynomial time.
It induces a set-valued semantics
\[
W_\Theta(x) \;=\; \{\, w \mid V_\Theta(x,w)=1 \,\}.
\]

\begin{definition}[Rule-Internal and Rule-External Procedures]
A \emph{rule-internal} algorithm is any probabilistic polynomial-time algorithm
whose access is restricted to the public description of $\Theta$ and evaluation
of $V_\Theta$, and whose output depends only on $x$ and internal randomness.

A \emph{rule-external} algorithm is an algorithm that additionally receives
auxiliary structure $t$ not derivable from the public verification semantics
(e.g.\ trapdoors, generative rules, or provenance information).
\end{definition}

Rule-internal procedures correspond to what an observer or verifier can do
within the accepted rules of the system; rule-external procedures represent
additional generative structure.

% ============================================================
\section{Verification vs Construction}

\begin{proposition}[Verification is Rule-Internal]
For any $(x,w)$, membership in $W_\Theta(x)$ is decidable in deterministic
polynomial time by evaluating $V_\Theta(x,w)$.
\end{proposition}

\begin{proposition}[Construction May Be Rule-External]
If for every PPT rule-internal algorithm $\mathcal{A}$,
\[
\Pr[\mathcal{A}(x)\in \mathcal{W}_\Theta(x)] \le \negl(|x|).
\]
then any procedure that constructs a witness in $W_\Theta(x)$ with non-negligible
probability must rely on rule-external structure.
\end{proposition}

These propositions isolate the core asymmetry: verification is closed under the
public rules, while construction may fundamentally require additional structure.

% ============================================================
\section{Interface Non-Invertibility}

We abstract verification interfaces as maps from hidden generative objects to
observer-accessible data.

\begin{definition}[Interface Map]
Let $\mathcal{S}$ be a space of underlying structures (states, proofs, or
witnesses) and $\mathcal{I}$ a space of observable data.
An interface is a map
\[
\Phi:\mathcal{S}\longrightarrow\mathcal{I}.
\]
The interface is \emph{invertible} if $\Phi$ is injective, and
\emph{non-invertible} otherwise.
\end{definition}

\begin{theorem}[Interface Non-Invertibility]
If the interface $\Phi$ factors through a projection whose image has strictly
lower dimension or information content than $\mathcal{S}$, then $\Phi$ is
non-invertible: distinct underlying structures induce identical observable
behavior.
\end{theorem}

\begin{proof}
Any such projection identifies multiple points in $\mathcal{S}$.
Thus $\Phi(s)=\Phi(s')$ for some $s\neq s'$, and $\Phi$ cannot be injective.
\end{proof}

This theorem is elementary but fundamental: it explains why verification
interfaces can be complete for admissibility while incomplete for generation.

% ============================================================
\section{Canonical Instances}

\subsection{Quantum Measurement Interfaces}

A quantum state $\rho$ is accessed via measurement operators
$\mathcal{M}=\{M_i\}$, yielding expectation values
\[
\Phi_Q(\rho)=\{\mathrm{Tr}(\rho M_i)\}_{i}.
\]

The interface $\Phi_Q$ is invertible if and only if $\mathcal{M}$ is
informationally complete.
In typical experimental settings it is not, and the measurement interface is
therefore non-invertible.
Quantum entanglement exemplifies this phenomenon: correlations are verifiable,
but the global generative structure cannot be reconstructed from the interface.

\subsection{PCP Local Verification}

A PCP proof $\pi\in\{0,1\}^N$ is accessed by a verifier that queries $O(1)$ bits
per random seed $r$.
The induced interface
\[
\Phi_P(\pi)=\{\Pr[\text{Verifier accepts with randomness } r]\}_{r}
\]
is non-invertible by an information-theoretic counting argument.
Distinct proofs induce identical local verification behavior.

\subsection{Cryptographic Hardness}

Public-key cryptography fits the same pattern.
Public rules define a verification-complete interface, while secret keys provide
rule-external structure enabling construction.
Hardness arises precisely from the non-invertibility of the public interface.

% ============================================================
\section{Interpretive Observation}

\begin{observation}
Across these domains, apparent hardness or non-classical behavior is not caused
by a lack of structure, but by the deliberate separation between rule-internal
verification interfaces and rule-external generative structure.
What cannot be reconstructed is not absent; it lies outside the accessible
interface.
\end{observation}

\begin{conjecture}[Internal--External Rule Separation as a Structural Principle]
Across multiple foundational theories—including computational complexity,
cryptography, proof systems, and quantum mechanics—there appears a recurring
structural distinction between rule-internal verification interfaces and
rule-external generative structure.

We conjecture that this separation is not accidental, but reflects a general
organizational principle: systems that admit efficient, stable verification
necessarily restrict access to their full generative structure, rendering the
verification interface non-invertible.

Under this view, distinctions such as $\mathbf{P}$ versus $\mathbf{NP}$,
public-key cryptography, proofs without recoverability, and quantum
entanglement arise as theory-specific manifestations of the same abstract
rule-separation phenomenon, instantiated at different levels of description.
\end{conjecture}

\begin{conjecture}[Observer Interface Non-Invertibility]
Let $\mathcal{S}$ be a space of underlying structures and let
$\mathcal{I}$ be a space of observer-accessible data.
Let
\[
\Phi : \mathcal{S} \longrightarrow \mathcal{I}
\]
be a map induced by a fixed rule system $\Theta$ governing admissible
observations.

Assume that $\Theta$ specifies a verification-complete interface in the sense
that, for any $s \in \mathcal{S}$, relational properties of $s$ expressible in
terms of $\mathcal{I}$ are decidable via $\Phi(s)$ using only the rules in
$\Theta$.

We conjecture that, generically, such observer interfaces are non-invertible:
there exist distinct $s, s' \in \mathcal{S}$ with
\[
\Phi(s) = \Phi(s').
\]

Equivalently, the rule system $\Theta$ induces an information-compressing
projection from $\mathcal{S}$ to $\mathcal{I}$, under which reconstruction of
the full generative structure from observer-accessible data is not possible.

Under this conjecture, limits on explanation arise from structural
non-invertibility of the observation interface rather than from lack of
underlying structure or insufficient computational power.
\end{conjecture}

\begin{conjecture}[Existence Without Recoverability under Rule-Constrained Interfaces]
Let $\mathcal{S}$ be a space of generative structures and $\mathcal{I}$ a space
of observer-accessible data.
Let
\[
\Phi:\mathcal{S}\longrightarrow\mathcal{I}
\]
be an interface map induced by a fixed rule system $\Theta$, and define the
fibers $\mathcal{S}_i=\Phi^{-1}(i)$.

We conjecture that, generically, such interfaces exhibit the following
NP-style phenomenon:

\begin{enumerate}[label=(\roman*)]
\item \emph{Existence.}
For every realizable interface datum $i\in\mathcal{I}_0:=\Phi(\mathcal{S})$,
there exists at least one $s\in\mathcal{S}$ such that $\Phi(s)=i$.
Equivalently, a selector (right-inverse) exists non-constructively on
$\mathcal{I}_0$.

\item \emph{Non-recoverability under rules.}
There is no rule-internal probabilistic polynomial-time procedure that can
uniformly obtain such a selector or recover a representative $s\in\mathcal{S}_i$
from $i$ with non-negligible success.
Recoverability fails relative to the rule system $\Theta$ and the interface
$\Phi$, despite existence.

\item \emph{Structured exceptions.}
Recoverability becomes feasible only upon augmenting the rule system with
additional rule-external structure (e.g., trapdoors or highly regular auxiliary
information) that effectively alters the interface.
Such exceptions are non-generic and correspond to deliberately engineered or
highly constrained regimes.
\end{enumerate}

This conjecture concerns recoverability \emph{relative to a fixed interface and
rule system}.
Non-recoverability is not a claim about non-existence of generative structure,
but about the structural impossibility of obtaining it through rule-internal
access alone.
\end{conjecture}

\paragraph{Why NP-style?}
The conjecture is NP-style because it separates \emph{existence} from
\emph{recoverability}.

In classical complexity theory, a language $L$ belongs to $\mathbf{NP}$ if
membership admits a witness whose validity is efficiently verifiable, even if
no efficient procedure is known for constructing such a witness.
The defining feature is not hardness of verification, but the gap between
existence and construction.

The interface setting exhibits the same structure.
For each observable datum $i\in\mathcal{I}_0$, existence asserts
\[
\exists\, s\in\mathcal{S}\quad \Phi(s)=i,
\]
while non-recoverability asserts that no rule-internal procedure can obtain such
an $s$ from $i$ with non-negligible success.
Thus the conjecture formalizes an $\exists$-statement without an accompanying
search procedure, directly mirroring the semantic core of $\mathbf{NP}$.

In this sense, the conjecture is NP-style not by complexity classification, but
by logical form: existence is guaranteed, verification is well-defined, and
construction is obstructed by the interface.

We also note the possibility that aspects of the generative structure may be
indirectly probed or constrained within domains accessible to observers,
without enabling full reconstruction through the observation interface.

% ============================================================
\section{Scope and Non-Claims}

This work does not propose new algorithms, cryptographic primitives, or physical
theories.
It does not resolve the P versus NP question, nor does it modify quantum
mechanics.
Its contribution is structural: it identifies a shared interface-level
obstruction underlying diverse no-go and hardness results, and provides a
uniform language in which they can be compared.

\end{document}
