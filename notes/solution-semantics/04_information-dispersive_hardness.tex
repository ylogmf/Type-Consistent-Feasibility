\documentclass[11pt]{article}

\usepackage{amsmath,amssymb,amsthm}
\usepackage{geometry}
\usepackage{enumitem}
\usepackage{hyperref}

\geometry{margin=1in}

% ---------- Environments ----------
\newtheorem{definition}{Definition}
\newtheorem{observation}{Observation}
\newtheorem{proposition}{Proposition}
\newtheorem{conjecture}{Conjecture}

% ---------- Macros ----------
\newcommand{\negl}{\mathrm{negl}}
\newcommand{\poly}{\mathrm{poly}}

\title{Information-Dispersive Hardness\\
under Rule-Constrained Interfaces}
\author{}
\date{}

\begin{document}
\maketitle

\begin{abstract}
We study a form of hardness that arises not from the non-existence of solutions,
but from the structure of the rule systems under which solutions are verified.
The central phenomenon, which we call \emph{information-dispersive hardness},
occurs when admissible operations are efficiently computable and verifiable
under a fixed rule system, yet no procedure permitted by those same rules can
recover the underlying generative structure.
This perspective separates existence from recoverability in an NP-style manner
and provides a unifying structural explanation for cryptographic hardness,
local verification systems, and related no-go phenomena.
\end{abstract}

% ============================================================
\section{Rule Systems as First-Class Objects}

We treat rules, rather than functions or algorithms in isolation, as the
primary objects of analysis.

\begin{definition}[Rule System]
A rule system $\Theta$ consists of:
\begin{itemize}
  \item a set of admissible objects $\mathcal{O}_\Theta$;
  \item a set of admissible operations $\mathcal{F}_\Theta$ executable under the
  rules;
  \item a verification predicate $\mathcal{V}_\Theta$ specifying admissibility.
\end{itemize}
\end{definition}

All notions of feasibility, computability, and recoverability in this work are
explicitly relative to a fixed rule system $\Theta$.
No operation is considered unless it is derivable from, or explicitly added to,
$\Theta$.

This shift is deliberate: hardness is not viewed as an intrinsic property of a
function, but as a relational property between an operation and the rules under
which it is executed.

% ============================================================
\section{Verification Semantics and Witness Sets}

Let $V_\Theta(x,w)\in\{0,1\}$ be a deterministic polynomial-time verification
predicate permitted by $\Theta$.
For each instance $x$, this induces a witness set
\[
W_\Theta(x)=\{\, w \mid V_\Theta(x,w)=1 \,\}.
\]

\begin{observation}
Verification under $\Theta$ has set-valued semantics:
it establishes membership in $W_\Theta(x)$, but does not identify a unique
witness.
\end{observation}

This observation is elementary but foundational.
In particular, the verification interface answers the question
``is this admissible?'', not ``how was this generated?''

% ============================================================
\section{Rule-Internal and Rule-External Procedures}

\begin{definition}[OWF-Induced NP Relation under a Rule System]
Let $\Theta$ be a rule system whose admissible operations
$\mathcal{F}_\Theta$ include a family of forward-efficient mappings
\[
g_\lambda : \{0,1\}^{n(\lambda)} \to \{0,1\}^{m(\lambda)},
\]
generated by a polynomial-time procedure.

Assume that inversion of $g_\lambda$ is infeasible for any probabilistic
polynomial-time rule-internal procedure under $\Theta$ (i.e., $g_\lambda$ is
one-way relative to $\Theta$).

Define the induced NP verification relation by
\[
R^{(g)}_\lambda(x,w) \;=\; \mathbf{1}[\, g_\lambda(w) = x \,],
\]
and the corresponding induced language by
\[
L^{(g)}_\lambda
\;=\;
\{\, x \mid \exists w\; R^{(g)}_\lambda(x,w)=1 \,\}
\;=\;
\mathrm{Im}(g_\lambda).
\]
\end{definition}


Rule-internal procedures describe what is possible within the verification
interface.
Rule-external procedures introduce additional generative capability by
augmenting the rule system.

This distinction is not about computational power, but about rule access.

\begin{definition}[Rule-Internal Constructibility]
Let $\Theta$ be a rule system and let
$f \in \mathcal{F}_\Theta$ be an admissible operation mapping witnesses to
instances.

We say that $f$ is \emph{rule-internally constructible relative to $\Theta$} if
there exists a probabilistic polynomial-time rule-internal procedure
$\mathcal{A}$ such that, for all admissible instances $x$ in the image of $f$,
\[
\Pr\bigl[f(\mathcal{A}(x)) = x\bigr] \ge \frac{1}{\poly(|x|)}.
\]
\end{definition}

% ============================================================
\section{Information-Dispersive Operations}

We now formalize the central notion of this work.

\begin{definition}[Information-Dispersive Operation]
Let $\Theta$ be a rule system and let
$f \in \mathcal{F}_\Theta$ be an admissible operation mapping witnesses to
instances.

We say that $f$ is \emph{information-dispersive relative to $\Theta$} if:
\begin{enumerate}[label=(\roman*)]
  \item $f$ is efficiently computable using only operations permitted by
  $\Theta$;
  \item consistency with respect to $f$ is decidable under the verification
  predicate $\mathcal{V}_\Theta$;
  \item no rule-internal procedure under $\Theta$ can invert $f$ with
  non-negligible probability.
\end{enumerate}
\end{definition}

\begin{observation}
Information-dispersive hardness is not a property of functions in isolation,
but of functions viewed relative to a fixed rule system.
\end{observation}

In this sense, the same function may be dispersive under one rule system and
recoverable under another.

% ============================================================
\section{Existence Without Recoverability}

We now state the central conjecture.

\begin{conjecture}[Existence Without Recoverability under Rule-Constrained Interfaces]
Let $\Phi:\mathcal{S}\to\mathcal{I}$ be an interface map induced by a rule system
$\Theta$, and define fibers $\mathcal{S}_i=\Phi^{-1}(i)$.

We conjecture that, generically, such interfaces exhibit the following NP-style
phenomenon:
\begin{enumerate}[label=(\roman*)]
  \item (\emph{Existence}) For every realizable interface datum
  $i\in\Phi(\mathcal{S})$, there exists at least one $s\in\mathcal{S}$ such that
  $\Phi(s)=i$.
  \item (\emph{Non-recoverability}) No rule-internal procedure under $\Theta$ can
  uniformly recover a representative $s\in\mathcal{S}_i$ from $i$ with
  non-negligible success.
  \item (\emph{Structured exceptions}) Recoverability becomes feasible only when
  the rule system is augmented with rule-external structure, such as trapdoors
  or highly regular auxiliary information.
\end{enumerate}

This conjecture concerns recoverability relative to a fixed rule system and
interface.
Non-recoverability does not imply non-existence of generative structure, but
reflects the structural limits imposed by rule-internal access.
\end{conjecture}

\paragraph{Why NP-style?}
The conjecture is NP-style because it separates existence from construction.
As in $\mathbf{NP}$, admissibility is verifiable given a witness, and existence
is asserted, while uniform recovery of such witnesses is obstructed by the
rules governing access.

% ============================================================
\section{Illustrative Domains}

We briefly indicate how the same structure appears across several domains.

\subsection{Cryptography}
Public-key cryptography exposes a verification-complete rule-internal interface,
while private keys supply rule-external structure enabling construction.
Hardness arises from the non-recoverability of generative structure under public
rules.

\subsection{PCP and Local Verification}
PCP systems allow global consistency to be verified via local checks.
The induced verification interface is information-theoretically incapable of
recovering the underlying proof, illustrating information-dispersive hardness
without computational assumptions.

\subsection{Quantum Measurement}
Quantum measurement provides access to statistical correlations via a fixed
measurement interface.
When the accessible measurements are not informationally complete, the mapping
from quantum states to observable statistics is non-invertible.
Entanglement exemplifies verification-complete but generation-incomplete
structure.

% ============================================================
\section{Scope}

This work does not propose new algorithms, cryptographic primitives, or physical
theories.
It does not resolve the P versus NP question, nor does it modify quantum
mechanics.
Its contribution is structural: it isolates a common interface-level obstruction
underlying diverse hardness and no-go phenomena.

\end{document}
