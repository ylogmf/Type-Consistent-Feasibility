\documentclass[11pt]{article}

\usepackage[margin=1in]{geometry}
\usepackage{amsmath, amssymb}
\usepackage[none]{hyphenat}

\title{Constructed NP Spaces:\\
Geometry, Scarcity, and Navigation}
\author{Yanlin Li}
\date{\today}

\begin{document}
\maketitle

\begin{abstract}
Classical complexity theory focuses on the classification of languages into
complexity classes such as $\mathbf{P}$ and $\mathbf{NP}$. This note adopts a
complementary perspective: instead of classifying problems, we study the
\emph{construction of feasibility spaces}. We argue that cryptography, artificial
intelligence, and game theory can be unified under a geometric interpretation of
$\mathbf{NP}$ as a space of admissible witnesses shaped by rules, encodings, and
verification interfaces. Hardness, from this viewpoint, is a property of the
constructed space rather than of individual algorithms.
\end{abstract}

\section{From Classification to Construction}

The classical $\mathbf{P}$ versus $\mathbf{NP}$ question asks whether a given
language belongs to one class or the other. Implicit in this formulation is the
assumption that the problem space is fixed and that the primary task is
classification.

In many practical and theoretical settings, however, the central activity is
different. Rather than asking whether a solution exists, practitioners
deliberately design problem spaces in which solutions are scarce, structured, or
strategically accessible. This motivates a shift in perspective: from the
classification of languages to the construction of feasibility spaces.

\section{NP as a Space of Witnesses}

Fix a parameter system $\Theta$ specifying encodings, syntactic constraints, and
admissibility rules. An $\mathbf{NP}$ problem under $\Theta$ induces a
set-valued semantics via its verification predicate $V_\Theta(x,w)$:
\[
\mathcal{W}_\Theta(x) := \{\, w \mid V_\Theta(x,w)=1 \,\}.
\]

The witness set $\mathcal{W}_\Theta(x)$ may be empty, finite, infinite, sparse,
structured, or degenerate. This suggests interpreting $\mathbf{NP}$ not merely as
a complexity class, but as a geometry over witness sets embedded in a larger
admissible space.

\section{RSA as a Near-Point Slice}

RSA-like constructions exemplify an extreme regime of this geometry. The ambient
admissible space is astronomically large, while the valid witness set effectively
collapses to a near-point slice, up to trivial symmetries.

From this viewpoint, RSA is not difficult merely because integer factorization is
hard, but because it engineers a feasibility space in which admissible witnesses
occupy an exceptionally small region. Cryptographic hardness thus appears as a
form of geometric scarcity rather than raw computational intractability.

\section{Cryptography as Space Engineering}

Modern cryptography can be reinterpreted as the deliberate construction of massive
$\mathbf{NP}$ spaces whose solution regions are vanishingly small. Design levers
include admissibility rules, encoding asymmetries, verification efficiency, and
controlled rule externalization.

Security, under this lens, is not a property of algorithms alone, but of the shape
of the constructed feasibility space and the interfaces through which it may be
explored.

\section{AI and Game Theory as Navigation Problems}

Where cryptography hides solutions, artificial intelligence and game theory
navigate them. Learning algorithms explore feasible regions, game-theoretic
agents reason over equilibrium sets, and planning systems traverse admissible
witness manifolds.

In this sense, AI does not ``solve'' $\mathbf{NP}$ problems in the classical
sense. Instead, it navigates constructed $\mathbf{NP}$ spaces, prioritizing
coverage, approximation, and strategic movement over isolation of unique
solutions.

\section{Scope and Non-Claims}

This work makes no claims regarding the resolution of the classical $\mathbf{P}$
versus $\mathbf{NP}$ question. It proposes no new cryptographic primitives and
introduces no algorithmic separations.

Its contribution is conceptual: to articulate a unifying geometric view in which
cryptography, artificial intelligence, and strategic reasoning are understood as
different modes of interaction with constructed $\mathbf{NP}$ spaces.

\section{Conclusion}

The classical question asks whether a solution exists. This note asks instead
what kind of solution space has been constructed, and who is able to move within
it. This shift reframes computational hardness as a property of space design,
rather than of isolated decision procedures.

\end{document}

\section*{Citation}

Li, Yanlin.
\emph{<TITLE>}.
Version~v0.4, 2025.
Available at: \url{https://github.com/ogmf/Type-Consistent-Feasibility/tree/v0.4}.

\noindent Please cite the version you consulted.

\bibliographystyle{plain}
\bibliography{references}
