\documentclass[sigconf,nonacm]{acmart}

\usepackage{amsmath, amssymb, amsthm}
\usepackage{geometry}
\usepackage{hyperref}
\usepackage{tabularx}
\usepackage[T1]{fontenc}
\usepackage{lmodern}
\usepackage{cmap}        % better copy/paste from PDF
\usepackage{microtype}   % better text shaping
\usepackage{ragged2e}
\usepackage{array} % 提供 \arraybackslash 和 \newcolumntype
\usepackage{float}      % for [H]
\usepackage{placeins}   % for \FloatBarrier (optional but useful)
\usepackage{mathtools} 


\geometry{margin=1in}

\title{I Love Peporoni Pizza}
\subtitle{Notes on Time, Interface, and Why None of This Prevents Dinner}
\author{Yanlin Li}
\affiliation{%
  \country{USA}
}
\email{yl@ogmf.net}

\date{\today}

\newtheorem{observation}{Observation}
\newtheorem{definition}{Definition}
\newtheorem{remark}{Remark}
\newtheorem{theorem}{Theorem}
\newtheorem{conjecture}[theorem]{Conjecture}

\newcolumntype{Y}{>{\raggedright\arraybackslash}X}

\begin{abstract}
This note records a sequence of observations concerning time, causality, and
quantum phenomena, framed through the concept of \emph{observer interfaces}.
Rather than advancing a closed physical theory, this note records a sequence of
structural observations concerning time, causality, and quantum phenomena,
framed through the concept of observer interfaces. Classical continuity, quantum discontinuity,
and apparent violations of causality are examined under a unified interface-based
perspective.
\end{abstract}

\begin{document}

\maketitle

\section{Introduction}

Many foundational debates in physics, computation, and philosophy revolve
around perceived limits: limits of speed, limits of knowledge, limits of
reversibility, and limits of decision.
These limits are often treated as intrinsic properties of reality itself.

This note adopts a different perspective.
We argue that many such limits are more fruitfully understood as consequences
of \emph{interfaces}—formal or informal contracts governing how systems interact,
observe, and verify—rather than as statements about what exists.

The observations presented here are modest in scope.
They do not propose new laws, nor do they attempt to resolve classical open
problems.
Their purpose is to make explicit a structural viewpoint that recurs across
multiple domains.

For the purposes of this note, an \emph{observer interface} refers to any
rule-constrained mapping that mediates between underlying generative states
and the observer-accessible records, queries, or measurements derived from them.

An \emph{observer interface} is a rule-constrained mapping
\[
\Phi : \mathcal{S} \to \mathcal{I},
\]
where \(\mathcal{S}\) denotes underlying generative states and
\(\mathcal{I}\) denotes observer-accessible records, measurements,
or descriptions.
Distinct states in \(\mathcal{S}\) may induce identical interface
data in \(\mathcal{I}\).

\noindent\textbf{Remark.}
No ontological claim is made regarding the nature of the generative states $\mathcal{S}$.
The interface formalism is agnostic to their physical or metaphysical status; it characterizes
only the rule-constrained mapping by which generative structure becomes observer-accessible.

\section{Time and Irreversibility as Interface Phenomena}

Time is commonly modeled as a fundamental dimension or parameter.
From the present perspective, time may instead be viewed as an emergent feature
of interface non-invertibility.

When an observer interacts with a system through an interface that compresses
information—discarding generative detail while preserving coarse consistency—
the resulting interaction becomes irreversible.
The ordering induced by this irreversible compression is experienced as time.

Under this view, temporal asymmetry does not require asymmetric microdynamics.
It arises whenever observation or interaction enforces a many-to-one projection
from underlying structure to accessible data.
\begingroup
\hyphenpenalty=10000
\exhyphenpenalty=10000
\begin{table*}[t]
\centering
\small
\setlength{\tabcolsep}{5pt}
\renewcommand{\arraystretch}{1.15}
\begin{tabularx}{\textwidth}{lYY}
\toprule
\textbf{Phenomenon} &
\textbf{Structural Cause (Interface Perspective)} &
\textbf{Effect of Interface Enhancement} \\
\midrule
Gravity &
Information compression across spatial degrees of freedom due to interface non-invertibility &
Effective attraction weakens; trajectories become less biased as finer spatial distinctions are resolved \\
Energy &
Interface occupancy density (cardinality of microstate equivalence classes) &
Effective mass or inertia decreases as degeneracy is resolved into distinguishable microstates \\
Speed of light limit &
Finite causal bandwidth of the observer interface &
Maximum signal speed increases only if interface bandwidth is fundamentally expanded \\
Arrow of time &
Irreversible information compression under observation &
Temporal asymmetry diminishes as reversible micro-dynamics become externally accessible \\
Quantum nonlocal correlations &
Projection of generative-level nonlocality through a constrained interface &
Correlations appear less paradoxical as intermediate generative structure becomes observable \\
No superluminal communication &
Inability to control or steer interface fibers &
Limited signaling may emerge only if the interface allows directed control of microstate selection \\
\bottomrule
\end{tabularx}
\caption{Physical phenomena interpreted as consequences of interface structure, together with their expected modification under enhanced interface resolution or access.}
\end{table*}

\endgroup

\section{Gravity and Black Holes as Interface-Induced Bias}

Gravity is traditionally treated either as a fundamental interaction
or as a manifestation of spacetime curvature.
From the present interface-based perspective, gravity may be interpreted
as an emergent bias induced by irreversible information compression
across spatial degrees of freedom.

Consider an observer interacting with a system through an interface
that maps a high-dimensional generative structure onto a lower-dimensional
observable description.
When distinct micro-configurations differing in spatial arrangement
are projected into equivalence classes under this interface,
the resulting dynamics exhibit a systematic drift toward configurations
of higher degeneracy.

This degeneracy-induced drift appears, at the interface level,
as an effective attraction.
Importantly, no intrinsic attractive force is required at the
generative level for this behavior to arise.
The apparent gravitational tendency reflects the asymmetry introduced
by many-to-one projection: more underlying states correspond to
``closer'' or ``lower-potential'' configurations than to finely separated ones.

Under this view, gravitational irreversibility is not fundamental,
but interface-induced.
The observer cannot reconstruct the discarded spatial distinctions,
and the induced bias becomes temporally stable.
This aligns with the empirical universality of gravitation:
any sufficiently coarse interface that compresses spatial microstructure
will induce a similar large-scale regularity.

\begin{observation}[Interface-Induced Gravity]
If an observer accesses spatial structure only through a
non-invertible interface \(\Phi\), then interface-level dynamics
exhibit a systematic aggregation bias toward configurations with
larger preimages under \(\Phi\).
This bias manifests as an effective attractive interaction.
\end{observation}

\begin{remark}
Gravity is unavoidable for observers not because attraction is fundamental,
but because any observer interface that irreversibly compresses spatial
information induces a systematic aggregation bias in the space of
accessible configurations.
\end{remark}

\subsection{Interface Enhancement and Gravitational Weakening}

If the observer interface were enhanced so as to retain finer spatial
distinctions—i.e., reducing the degree of compression—the effective
gravitational bias would weaken.
Trajectories that appear strongly curved under a coarse interface
would appear progressively less constrained as additional degrees of
freedom become externally accessible.

In the limiting case of a fully invertible spatial interface,
no gravitational attraction would be observed at all.
Dynamics would be described directly in terms of generative structure,
without emergent bias induced by information loss.

This interpretation does not contradict general relativity.
Rather, it reframes curvature as an interface-stable description of
systematic compression effects, leaving the underlying generative
ontology unspecified.

The phenomena described above characterize observer-level capacities
and constraints imposed by a fixed interface.
They are not independent physical forces, but structural regularities
induced by limitations on information access, control, and reversibility.
The four fundamental interactions discussed next arise as stable
instantiations within these interface-imposed constraints.

\subsection{Black Holes as the Extreme Limit of Interface Compression}

From the interface-based perspective developed above, black holes may be
understood as the extreme regime of gravitational phenomena, corresponding
to maximal interface-induced compression of spatial and generative
information.

In ordinary gravitational settings, interface non-invertibility induces a
systematic bias in observable trajectories by collapsing distinct spatial
micro-configurations into coarse equivalence classes.
A black hole represents the limiting case in which this compression becomes
so severe that almost all generative distinctions beyond a small set of
global invariants are projected away.

At the interface level, this manifests as an effective loss of retrievable
structure.
Observable quantities converge toward a minimal description, while the
cardinality of the underlying equivalence classes grows without bound.
This behavior does not require the absence of underlying generative
information, but follows from the observer interface becoming maximally
non-invertible.

In this sense, a black hole is not characterized here as a breakdown of
physical law, but as an interface-saturating gravitational configuration.
Gravitational attraction and information compression coincide: gravity
reaches a regime where further spatial distinctions no longer survive
projection to the observable interface.

This interpretation is compatible with existing physical theories.
It does not assert new dynamical mechanisms, nor does it modify relativistic
or quantum descriptions.
Rather, it classifies black holes as boundary cases in which gravitational
effects and interface compression become inseparable.

This interpretation is classificatory rather than mechanistic: it concerns
interface saturation and information compression, not horizon microphysics,
evaporation dynamics, or specific proposals for information recovery.

\section{The Four Fundamental Interactions as Interface-Conditioned Regularities}

This section is organized around Table~2, which provides a compact
interface-level synopsis of the four fundamental interactions.
The surrounding text serves only to clarify the structural reading
encoded in the table.

We record a compact interface-level synopsis of the four fundamental interactions.
The intent is not to reduce physical theory to a single principle, but to express
a common structural reading: observable regularities may be shaped by the
information retained (and discarded) by an observer interface.

\begin{table*}[t]
\centering
\small
\setlength{\tabcolsep}{5pt}
\renewcommand{\arraystretch}{1.15}
\begin{tabularx}{\textwidth}{lYYY}
\toprule
\textbf{Interaction} &
\textbf{Interface-Retained Structure} &
\textbf{Compression Character} &
\textbf{Characteristic Interface Behavior} \\
\midrule
Gravitational &
Scalar occupancy measures reflecting total degeneracy across spatial degrees of freedom &
Fully scalar, sign-erasing compression over spatial configuration space &
Universally attractive and unscreenable; manifests as a global bias acting on all energy-localized interface states \\
Electromagnetic &
Signed and vector-valued relational attributes (e.g., charge and orientation) &
Partial compression preserving sign and directional information &
Attractive or repulsive; screenable; long-range correlations supported by retained vector structure \\
Weak &
Chiral and flavor-asymmetric admissibility labels &
Selective, symmetry-breaking compression over transition pathways &
Short-range and non-universal; induces irreversible state transmutations at the interface level \\
Strong &
Multi-component relational constraints governing compositional admissibility &
Minimal compression of internal relational structure; separability is suppressed &
Confining and saturating; prevents stable interface representation of isolated constituents \\
\midrule
\bottomrule
\end{tabularx}
\caption{The four fundamental interactions interpreted through interface retention and information compression, with a unified structural interpretation across physical, logical, and mathematical layers.}
\end{table*}

\noindent\textbf{Unified Interpretation.}
Across all four interactions, observed regularities arise from interface-constrained projections
of underlying generative structure. Physics characterizes invariant appearances under shared
observation interfaces; logic governs admissible inference and verification within those interfaces;
mathematics provides a consistency-based meta-language extending beyond observational regimes
without asserting recoverability or physical realization.


The phenomena listed above characterize observer-level capacities and constraints
imposed by a fixed interface.
They are not independent physical forces, but structural regularities induced by
limitations on information access, control, and reversibility.
The four fundamental interactions discussed next arise as stable instantiations
within these interface-imposed constraints.

\section{On Non-Prevention of Dinner}

None of the preceding observations impose practical constraints on ordinary
activities.
The existence of interface-induced irreversibility does not prevent meals,
choices, or schedules.
It merely explains why some processes cannot be undone, replayed, or perfectly
predicted.

Dinner remains decidable.

\section{Scope}

This note makes no claims regarding the resolution of open problems in physics or
theoretical computer science.
It introduces no new complexity classes, physical constants, or metaphysical
commitments.

Its contribution is interpretive.
By treating time, hardness, and irreversibility as interface phenomena, it offers
a unified lens through which otherwise disparate limits may be compared.

\section{Broader Implications of Interface Constraints}

\subsection{On Comparing ``All of P'' and ``All of NP'' Across Parameter Systems}

A common informal move is to compare ``the set of all languages in $P$'' with
``the set of all languages in $NP$,'' or to compare their respective
subcollections.  From the present parameterized/interface viewpoint, such
comparisons are not well-formed unless the underlying parameter system is fixed.

\paragraph{Parameter systems and induced universes.}
Let $\mathsf{T}$ be a class of admissible parameter systems.  For each
$\Theta \in \mathsf{T}$, let
\[
\Theta = (\Sigma_\Theta,\mathrm{Syn}_\Theta,\mathrm{Sem}_\Theta,\mathrm{Prag}_\Theta),
\qquad
U_\Theta := \Sigma_\Theta^* .
\]
A (classical) language under $\Theta$ is an object
\[
L \subseteq U_\Theta .
\]
Define the feasibility classes relative to $\Theta$ as
\[
P_\Theta \;:=\;\{\, L \subseteq U_\Theta \mid L \in P \text{ (over }U_\Theta\text{)} \,\},
\]
\[
NP_\Theta \;:=\;\{\, L \subseteq U_\Theta \mid L \in NP \text{ (over }U_\Theta\text{)} \,\}.
\]
Note that $P_\Theta$ and $NP_\Theta$ are sets of languages, hence
\[
P_\Theta,\, NP_\Theta \subseteq \mathcal{P}(U_\Theta),
\]
where $\mathcal{P}(\cdot)$ denotes the power set.

\paragraph{Disjoint-union packaging.}
To even speak about ``all'' feasible languages across varying $\Theta$ without
silently identifying different universes, the correct global objects are
:
\begin{align*}
\mathsf{P}^{\star} &:= \{\, (\Theta, L) \mid \Theta \in \mathsf{T}, L \in P_\Theta \,\}, \\
\mathsf{NP}^{\star} &:= \{\, (\Theta, L) \mid \Theta \in \mathsf{T}, L \in NP_\Theta \,\}.
\end{align*}

An element of $\mathsf{P}^\star$ is a pair $(\Theta,L)$ with $L\subseteq U_\Theta$,
not a language in a single fixed universe.

\begin{proposition}[Ill-typed global comparison]
Without fixing a single parameter system $\Theta$, statements of the form
``compare all of $P$ to all of $NP$'' are not well-formed as feasibility
comparisons of languages.  In particular, expressions such as
\[
\bigcup_{\Theta\in\mathsf{T}} P_\Theta \;\subseteq\; \bigcup_{\Theta\in\mathsf{T}} NP_\Theta
\]
do not have a canonical interpretation, because the left and right sides do not
denote languages in a fixed universe, and the underlying universes $U_\Theta$
need not coincide.
\end{proposition}

\begin{proof}
There are two independent typing obstructions.

\smallskip
\noindent\textbf{(1) Universe mismatch.}
If $\Theta_1,\Theta_2\in\mathsf{T}$ have different alphabets (or different
well-formedness regimes), then $U_{\Theta_1}=\Sigma_{\Theta_1}^*$ and
$U_{\Theta_2}=\Sigma_{\Theta_2}^*$ are different underlying universes.
Languages under $\Theta_1$ are subsets of $U_{\Theta_1}$, whereas languages under
$\Theta_2$ are subsets of $U_{\Theta_2}$.  Without an explicit identification map
between $U_{\Theta_1}$ and $U_{\Theta_2}$, there is no canonical meaning to
set-theoretic operations that silently mix elements from $\mathcal{P}(U_{\Theta_1})$
and $\mathcal{P}(U_{\Theta_2})$.

\smallskip
\noindent\textbf{(2) Object-level mismatch.}
Even if all $\Theta$ share the same alphabet so that $U_\Theta = U$ is fixed,
each $P_\Theta$ and $NP_\Theta$ is a \emph{set of languages}:
\[
P_\Theta,NP_\Theta \subseteq \mathcal{P}(U).
\]
Thus $\bigcup_{\Theta} P_\Theta$ and $\bigcup_{\Theta} NP_\Theta$ are also sets of
languages, i.e.\ subsets of $\mathcal{P}(U)$, not languages (subsets of $U$).
Feasibility predicates such as ``$\in P$'' and ``$\in NP$'' apply to languages
$L\subseteq U$, not to arbitrary collections of languages.

\smallskip
Therefore, without fixing $\Theta$ (and hence the subject language object), global
comparisons that range over varying parameter systems conflate distinct objects and
lack a well-typed feasibility interpretation.  Meaningful comparison arises only
after fixing a single $\Theta$, in which case the comparison is between properties
of the same induced language universe.
\end{proof}


\subsection{Hardness and the Separation of Verification and Construction}

For an observer, the world guarantees only verifiable existence, not recoverable truth.
All access to phenomena is mediated through finite and non-invertible observation interfaces.
Such interfaces allow verification of consistency and admissibility relative to fixed rules,
but they do not, in general, permit reconstruction of the underlying generative structure.

Observation therefore establishes existence in a relational sense: there exists at least one
underlying configuration compatible with the observed data. However, the interface does not
guarantee that this configuration is uniquely identifiable, reconstructible, or even describable
from within the interface itself. Multiple distinct generative structures may induce identical
observational outcomes.

Accordingly, truth understood as a uniquely recoverable, observer-independent object is not
promised by observation. What remains well-defined is verification: the ability to determine
whether an observation is consistent with the governing rules of the interface. This distinction
separates epistemic validity from ontological completeness and clarifies the limits of observer-based
knowledge.

\subsection{Mathematics Beyond Observational Regimes}

Mathematics is itself a formal language, but it is not an observational one.
Unlike empirical or physical descriptions, mathematics does not require observability,
constructibility, decidability, or interaction. Its foundational requirement is internal
consistency.

By relinquishing all interface-bound commitments, mathematics admits constructions that extend
far beyond any observational regime. For example, the collection of all formal languages over a
finite alphabet, defined set-theoretically as the power set of $\Sigma^*$, is mathematically
well-formed despite the fact that most such languages are undecidable, undefinable, and
inaccessible to any observer.

Mathematics therefore functions as a meta-language: it does not operate within particular
languages, but treats languages themselves as abstract objects. Its expressive scope arises not
from increased descriptive power, but from minimal semantic obligation. Mathematics can quantify
over structures without asserting their observability, realizability, or physical relevance.

\subsection{Physics as Interface-Constrained Description}

Physical laws do not describe reality in itself, but the stable invariants that persist under
a given mode of observation. They characterize how phenomena appear when filtered through
specific measurement interfaces governed by fixed rules.

A physical law is thus an interface law: a compressed description of regularities that remain
invariant across observers sharing comparable interfaces. The apparent universality of such laws
does not arise from direct access to generative structure, but from the structural similarity of
the observational constraints.

Phenomena such as gravity illustrate this point. Gravitational behavior is unavoidable for
observers operating within spacetime-based interfaces, not because gravity exhausts the nature
of reality, but because it emerges inevitably under interface-level compression. Extreme cases,
including black holes, correspond to limiting regimes of such compression rather than privileged
ontological disclosures.

\section{Self Note on Position and Discipline}

This work is written from the position of an observer rather than a claimant of privileged
insight. No assertion herein presumes access to truth beyond what is afforded by rule-governed
observation.

The universe is treated as a product of rules and structure rather than as a process with a
terminal state. Nothing in this framework implies culmination or finality.

Whether these ideas are understood, accepted, or rejected by others is structurally irrelevant.
Verification does not depend on consensus, only on internal consistency relative to declared
rules.

A distinction is maintained between intuition and justification. Subjective conviction or
experiential intensity is acknowledged but not granted epistemic authority. What may feel present
is not thereby claimed as possessed. Rational restraint is preserved explicitly to prevent such
slippage.

\section{Conclusion: None of This Prevents Dinner}

Interfaces shape what can be known, reversed, or constructed.
They do so quietly, without announcing themselves as fundamental laws.

The title of this note should not be overinterpreted.
Neither should reality.

The unifying interface-based perspective presented here should not be read
as a claim that these phenomena share a single underlying mechanism.
Rather, the claim is structural: diverse physical regularities may arise
whenever dynamics are necessarily expressed through limited, non-invertible
observer interfaces.

\section*{Recommended Citation}

\noindent\textbf{Recommended citation.}
Li, Yanlin.
\textit{I Love Peporoni Pizza: Notes on Time, Interface, and Why None of This Prevents Dinner},
Independent research note, 2025.

Available at: \url{https://github.com/ylogmf/Type-Consistent-Feasibility/tree/main/notes/solution-semantics/i_love_peporoni_pizza.pdf}.

\end{document}
