\documentclass[11pt]{article}

\usepackage[margin=1in]{geometry}
\usepackage{setspace}
\usepackage{amsmath,amssymb}
\usepackage{booktabs}
\usepackage{longtable}
\usepackage{hyperref}
\usepackage{booktabs}
\usepackage{tabularx}
\usepackage[authoryear]{natbib}

\onehalfspacing

\title{Observation as Irreversible Compression:\\
Rethinking the Limits of Physical Explanation\\
and the Cost of Formation}

\author{Yanlin Li}
\date{}

\begin{document}
\maketitle

\begin{abstract}
A difficulty arises when structural limits on physical
explanation stem from the irreversible roles of observation and
formation. It argues that many explanatory failures—particularly attempts
to reconstruct unique histories, ultimate origins, or generative
pathways—are not epistemic shortcomings, but consequences of irreversible
compression imposed by observational interfaces.

Within this framework, physical laws are interpreted as constraints on
admissible structure rather than encodings of realized history. Formation
incurs irreversible cost, eliminating generative distinctions that cannot
be recovered through observation, while time functions as a marker of such
expenditure rather than a coordinate of optimization or search.

The analysis explains why stable structures often appear optimized
without invoking generative enumeration or purposive arrangement.
Stability is shown to arise from survivability under interaction, not
traversal of alternatives. The paper further introduces stability
constraints governing cognition, decision, and observation across layered
systems, emphasizing that explanation identifies structural limits without
conferring privileged vantage points or normative authority.
\end{abstract}

\section{Introduction}
% Section 1

Physics is often characterized as the search for laws governing the
behavior of the natural world. Yet alongside this ambition lies a
persistent tension: not all questions framed as demands for explanation
admit completion, even in principle. In particular, attempts to
reconstruct unique histories, ultimate origins, or generative pathways
frequently encounter limits that cannot be overcome by improved
measurement, increased computational power, or more comprehensive
theories.

This paper argues that a substantial class of such limits is structural
rather than epistemic. These limits do not arise from ignorance,
approximation, or contingent technological constraints, but from the role
that observation itself plays in the formation of physical description.
Observation is not a passive window onto pre-existing detail; it functions
as an interface that irreversibly compresses generative processes into a
reduced set of distinguishable outcomes. Once such compression occurs,
certain distinctions are not merely hidden but eliminated at the level of
representation~\citep{Bell1990,Bohr1958}.

A recurring source of confusion in both physics and the philosophy of
science stems from the conflation of three distinct notions: rules,
formation, and observation. Physical laws constrain the class of admissible
structures; formation refers to the processes by which particular structures
come to be; observation registers only those aspects of formed structures
accessible through a given interface. Although these distinctions are
often acknowledged implicitly, they are frequently blurred in discussions
of irreversibility, information, time, and explanation. As a result,
failures of reconstruction are commonly misinterpreted as provisional gaps
in knowledge rather than as consequences of irreversible compression
\citep{Simon1996,Wittgenstein1953}.

Within the framework developed here, information is not identified with
semantic content, formal entropy, or physical substance, but with residual
distinguishability preserved under observational access. Formation, by
contrast, incurs irreversible cost: not all generative distinctions
survive into observable structure. Time is therefore treated not as a
fundamental coordinate of optimization, search, or reconstruction, but as
a marker of irreversible expenditure associated with formation.

This perspective clarifies several persistent explanatory temptations.
Stable physical structures often appear optimized, designed, or selected
from a space of possibilities. The present analysis rejects the inference
that such appearance implies generative search, purposive arrangement, or
enumerative optimization. Stability does not arise from the traversal of
alternatives, but from survivability under interaction. Unrealized
possibilities are not rejected options, but paths that never incurred the
irreversible cost of formation.

Crucially, explanatory appeal to stability carries no normative
implication. To explain why a structure persists is not to justify its
alteration, exploitation, or domination. Stability accounts for existence,
not authorization. This distinction is not an ethical supplement, but a
constitutive boundary of the explanatory framework itself.

Human observers and agents occupy no privileged position with respect to
this boundary. Humanity is a variable within a stable physical world, not
its culmination. The capacity to intervene in stable structures entails
responsibility rather than exemption. Where intervention exceeds the
tolerance of surrounding stability, systemic response follows—not as
moral judgment, but as structural consequence.

The aim of this paper is therefore deliberately limited. It does not propose new physical laws, revise established formalisms, or advance metaphysical claims about ultimate origins. Its contribution is to clarify the limits of explanation imposed by irreversible formation and observation, to distinguish stability from authorization, and to delineate where explanatory demands cease to increase structural distinction.

The structure of the paper is as follows. Section 2 develops the
conceptual framework, characterizing observation as irreversible
compression and defining information as residual distinguishability.
Section 3 examines the resulting limits on reconstruction and clarifies
why rules constrain possibility without encoding history. Section 4
explores the interpretation of time as formation cost. Section 5
addresses unrealized possibilities without invoking ontological
proliferation. Section 6 introduces stability constraints across cognitive,
decisional, observational, and runtime layers. The concluding section
summarizes the implications for explanation, stability, and restraint.

\section{Information, Observation, and Irreversible Formation}
% Section 2
This section develops the conceptual framework on which the subsequent
analysis depends. Its purpose is not to introduce new physical principles,
but to clarify the structural roles played by observation, formation, and
stability in delimiting what can and cannot be explained. Throughout, the
emphasis is placed on distinguishing structural constraints from epistemic
limitations, and descriptive explanation from normative implication.
\subsection{Observation as Interface-Dependent Compression}
% Subsection 2.1
Any physical observation presupposes an interface: a measurement
procedure, interaction channel, or representational scheme through which
physical processes become distinguishable. Such interfaces necessarily
implement many-to-one mappings. Distinct generative processes may yield
observationally identical outcomes when accessed through the same
interface.

Observation, in this sense, is not a passive revelation of pre-existing
detail, nor merely an imperfect approximation to an underlying
microdescription. It is a constitutive operation that irreversibly
compresses generative structure into a reduced representational form. Once
an observation is made, distinctions that are not preserved by the
interface do not remain available for recovery at that level of
description.

This irreversibility is structural rather than practical. It does not
depend on finite resolution, technological limitations, or insufficient
data. Even an idealized observer, endowed with unlimited memory and
computational capacity, would remain subject to the same constraint so long
as the observational interface is fixed. What the interface does not retain
is not hidden information awaiting extraction, but structure that no longer
exists as a distinguishable element of the observable description.
\citep{Zeh2007,Prigogine1980}

Observation thus introduces an asymmetry between formation and access.
Formation may generate a rich space of micro-differences, while observation
collapes this space into equivalence classes compatible with the
interface. The loss incurred is not epistemic uncertainty but
representational elimination.

\subsection{Information as Residual Distinguishability and Ontological Erasure}
% Subsection 2.2

Formation refers to the processes by which a system acquires a determinate
structure from a space of admissible possibilities. Such processes
necessarily involve interaction, constraint, and dissipation. Not all
distinctions present during formation survive into the formed structure.

Crucially, formation is not a computational procedure. It does not
enumerate alternatives, evaluate candidates, or select outcomes according
to an optimization criterion. Rather, formation is an irreversible physical
process in which certain configurations persist under interaction while
others do not. Unrealized possibilities are not rejected options; they are
paths that never incurred the irreversible cost required for realization.
\citep{Anderson1972}

Once formation has occurred, the generative distinctions eliminated during
the process cannot be reconstructed from the resulting structure. This
limitation does not arise from ignorance of governing rules, incomplete
access to initial conditions, or insufficient modeling power. It arises
because formation itself does not preserve the information required to
distinguish among alternative generative paths that converge on the same
observable outcome.

Accordingly, the presence of a stable structure should not be interpreted
as evidence of an underlying generative search, optimization, or design.
Stability reflects survivability under interaction, not selection over a
pre-evaluated space.


\subsection{Formation, Irreversibility, and the Role of Time}
% Subsection 2.3

Stability, as used in this framework, denotes the capacity of a structure
to persist under perturbation within a given set of constraints. Stable
structures exhibit robustness to variation and resistance to dissolution,
allowing them to endure and enter observational access.

The explanatory appeal to stability, however, is strictly descriptive. To
explain why a structure persists is not to justify its alteration,
exploitation, or domination. Stability accounts for existence; it does not
confer entitlement.
\citep{Simon1996,Wittgenstein1953}

This distinction constitutes a boundary condition of the framework. Any
account that invokes stability as an explanatory principle while deriving
normative authorization from persistence alone exceeds the scope of
analysis presented here. Stability explains why something remains; it does
not license what may be done to it.

The appearance of "design," optimization, or purposiveness associated
with stable structures arises from survivorship bias under irreversible
formation and observation. Structures that are insufficiently stable do not
persist long enough to be observed, described, or theorized. The resulting
world of observables therefore exhibits an intrinsic bias toward
robustness, regularity, and apparent order. This appearance should not be
misinterpreted as evidence of intentional arrangement or justificatory
priority.

\subsection{Information, Transmission, and Understanding}
% Subsection 2.4
Within the present framework, information is defined neither as semantic
content nor as a formal entropy measure, but as residual distinguishability
preserved under a given observational interface. Information consists of
those distinctions that survive irreversible compression and remain
accessible for verification, transmission, or coordination.

This definition emphasizes the relational character of information. What
counts as information depends on the interface through which a system is
accessed. Distinctions eliminated during formation or observation do not
persist as latent variables awaiting recovery; they are absent at the level
of representation.

This asymmetry explains a familiar but often misunderstood phenomenon:
information can be copied, transmitted, and verified with high fidelity,
while formation histories cannot. The rapid propagation of information
should not be conflated with the accumulation of understanding.
Understanding requires internal reconstruction of generative capacity,
which itself incurs irreversible formation cost.


\subsection{Structural Limits on Reconstruction}
% Subsection 2.5
The foregoing considerations motivate a reinterpretation of time within
the present framework. Time is not treated as a fundamental coordinate
along which systems evolve, but as a marker of irreversible expenditure
incurred through formation. What distinguishes earlier from later states is
not their position along an abstract axis, but the accumulation of
generative distinctions that can no longer be recovered.

This interpretation does not deny the operational utility of temporal
parameters in physical theories. Rather, it distinguishes between time as a
modeling coordinate and time as an explanatory indicator of irreversibility.
Coordinate time orders events; formation cost marks the boundary between
what can be reconstructed and what has been irretrievably lost.

\subsection{Scope and Clarifications}
% Subsection 2.6
The framework does not propose new physical laws, revise existing
formalisms, or adjudicate metaphysical questions concerning ultimate
origins. Its purpose is to fix conceptual distinctions necessary for the
analysis that follows, and to delineate the boundaries within which
explanation remains structurally meaningful.

In particular, the framework insists on three constraints: observation
entails irreversible compression; formation eliminates generative
distinction; and stability carries no normative authorization. These
constraints are inseparable. To remove any one of them is to alter the
scope of the analysis and invite misinterpretation.

Table~\ref{tab:structural_levels} should be read as a structural taxonomy
rather than a process model. Preservation at a given level does not imply
normative priority, authorization, or control. In particular, the
emergence of understanding does not constitute transcendence of the system
described, but introduces a high-impact variable subject to systemic
reaction. Stability explains persistence; it does not confer entitlement.

\bigskip

\noindent
\textbf{Table 1. Structural roles of information, formation, and time}

\bigskip

% Table 1 explanatory paragraph here

\begin{table}[t]
\centering
\small
\setlength{\tabcolsep}{6pt}
\renewcommand{\arraystretch}{1.2}
\begin{tabularx}{\textwidth}{lXXXX}
\toprule
\textbf{Level} &
\textbf{What is preserved} &
\textbf{What is eliminated} &
\textbf{Irreversibility source} &
\textbf{Explanatory role} \\
\midrule
Rule space &
Admissible structures and constraints &
No specific history &
None (possibility only) &
Defines what \emph{can} exist \\
Formation &
One realized path &
Alternative paths &
Physical formation cost &
Commits to realization \\
Observational interface &
Residual distinguishability &
Path-level history &
Interface compression &
Enables verification \\
Information transmission &
Stable distinctions &
Formation detail &
Copying fidelity &
Enables coordination \\
Understanding &
Reconstructed capacity &
Original history &
Internal re-formation &
Enables limited intervention \\
\bottomrule
\end{tabularx}
\caption{Structural roles of preservation, elimination, and irreversibility across descriptive levels.
The table should be read as a structural taxonomy rather than a process model.}
\label{tab:structural_levels}
\end{table}
\citep{Simon1996}


\section{Limits of Reconstruction}
% Section 3
The framework developed in the preceding section yields a direct and
unavoidable consequence: reconstruction of unique generative history from
observational data is, in general, impossible in principle. This limitation
does not arise from insufficient data, finite resolution, computational
intractability, or incomplete theoretical knowledge. It follows from the
structural role of observation and formation as irreversible processes.

\subsection{Rules Constrain Possibility, Not Realization}
% Subsection 3.1
Physical laws specify constraints on admissible structures and transitions.
They delimit what may exist and how transformations may occur. What they do
not encode is which admissible path was realized. Multiple generative
histories may satisfy the same rules and culminate in observationally
indistinguishable outcomes.

This distinction is often obscured by the predictive success of physical
theories. When laws reliably map initial conditions to final states, it is
tempting to treat them as implicitly containing the realized history.
However, such mappings presuppose access to initial conditions that are
themselves products of prior formation. Once these conditions are accessed
only through observation, the same irreversible compression applies
recursively.

Accordingly, laws function as validators rather than narrators. They
certify admissibility and consistency; they do not record contingency. To
mistake constraint for history is to misidentify the explanatory role of
law.
\citep{Carr2007}

\subsection{Structural Underdetermination of Histories}
% Subsection 3.2
Because observation implements many-to-one mappings, present states
accessed through a fixed interface correspond to equivalence classes of
generative histories. This underdetermination is structural rather than
epistemic. It does not depend on missing information, imperfect
measurement, or lack of computational power.

Even an ideal observer endowed with unlimited resources would face the
same limitation. Given a present observable state, no further observation
made through the same interface can discriminate among generative paths
that have already been compressed into equivalence. To recover such
distinctions would require an interface that preserves them—an interface
incompatible with the observation that has already occurred.

Thus, reconstruction fails not because the problem is difficult, but because the relevant distinctions no longer exist at the representational level.

\subsection{Minimal Illustrations of Irrecoverability}
% Subsection 3.3
The structural nature of reconstruction failure can be illustrated without
appeal to domain-specific physics.

Consider an arithmetic verification: the equality $1 + 1 = 2$ certifies
correctness within a rule system. The result "2" preserves admissibility
but does not encode the generative process by which it was obtained.
Multiple constructions are compatible with the same verified outcome. The
loss of generative path is not a matter of computational difficulty; it is
a consequence of representation. Verification erases history.

A similar asymmetry appears in biological systems. Genetic rules constrain
viable developmental outcomes, but they do not encode the unique sequence of
cellular interactions, environmental contingencies, and perturbations that
produced a particular organism. Possession of generative rules does not
entail access to generative biography.

These examples are not analogies intended to reduce physics to arithmetic or
biology. They are demonstrations of a shared structural feature: once
outcomes are certified under irreversible compression, history is no longer
represented.
\subsection{The Illusion of Completeness}
% Subsection 3.4
The belief that a sufficiently complete theory would permit full
reconstruction rests on a tacit assumption that observation is reversible
in principle. Once this assumption is abandoned, the expectation dissolves.

A "final theory" may fully characterize admissible structures, symmetries,
and transformations while leaving realized histories underdetermined. This
does not signal explanatory failure. It signals a mismatch between
explanatory demand and structural possibility.

Explanations account for regularity and constraint. Histories account for
contingency. When explanation is asked to deliver history, it is asked to
produce distinctions that have already been eliminated by formation and
observation.

\subsection{Reconstruction at the Scale of the Universe}
% Subsection 3.5
When the foregoing reasoning is extended to the universe treated as a
single system, the same structural limitation applies. Any observational
access to the universe yields a compressed description consistent with the
observational interface. The generative history of the universe—understood
as a unique path through admissible possibilities—cannot be reconstructed
from such access.

This claim does not assert that the universe lacks a history, nor that its
origin is metaphysically inaccessible. It asserts only that origin
narratives cannot be recovered from observational residues once
irreversible formation has occurred. The limitation is structural, not
theological.

Attempts to evade this conclusion by invoking ever finer reconstruction,
external observers, or implicit generative searches misconstrue the role of
observation. Refinement of description does not restore eliminated
distinctions.

\subsection{Structural, Not Epistemic, Limits}
% Subsection 3.6
The limits identified here are structural rather than epistemic. They
persist regardless of theoretical refinement, experimental sophistication,
or computational advance. Their source lies in the asymmetry between
formation and observation: formation generates distinctions that
observation does not preserve.

Recognizing this asymmetry clarifies a wide range of persistent confusions
in discussions of determinism, completeness, and explanation. Certain
questions fail not because answers are unknown, but because the
distinctions they presuppose no longer exist.

This recognition does not diminish the explanatory power of physics. On the
contrary, it sharpens that power by aligning explanatory ambition with
structural possibility. Explanation ends where distinction ends—not in
mystery, but in form.

\section{Time as Irreversible Cost, Not Fundamental Coordinate}
% Section 4
The structural limits on reconstruction established in the preceding
sections invite a reconsideration of the role time plays in physical
explanation. In standard physical formalisms, time appears as a coordinate
parameter along which systems evolve according to dynamical laws. This
representation is indispensable for prediction and calculation. However,
when time is treated exclusively as a coordinate, it obscures a distinct
explanatory role that emerges once irreversible formation and observation
are taken seriously.

Within the present framework, time is not introduced as an additional
substance, dimension, or primitive ordering principle. Rather, time is
interpreted as a marker of irreversible cost incurred through formation.
What distinguishes earlier from later states is not their position along an
abstract temporal axis, but the accumulation of generative distinctions
that can no longer be recovered.
\citep{Zeh2007,Prigogine1980}

\subsection{Irreversibility and the Direction of Time}
% Subsection 4.1
Irreversibility is often associated with the arrow of time, commonly
explained through entropy increase under coarse-graining. While this
account successfully captures macroscopic temporal asymmetry, it is
frequently interpreted as an epistemic artifact of incomplete description.
The framework developed here identifies a more fundamental source of
irreversibility: observation and formation themselves.

As argued in Section 2, observation irreversibly compresses generative
processes into reduced representational forms. Once such compression occurs,
distinctions among alternative generative paths are eliminated rather than
concealed. The directionality associated with time reflects the
accumulation of these eliminations. Later states are those from which fewer
generative distinctions can be recovered.

On this view, temporal asymmetry does not arise from a preferred direction
imposed by laws, but from the asymmetry between formation and access. Laws
may be time-symmetric; formation is not.
\subsection{Time Without a Reversible Axis}
% Subsection 4.2
Interpreting time as cost does not deny the operational utility of temporal
coordinates in physical theories. Instead, it distinguishes between time as a
modeling parameter and time as an explanatory boundary.

Coordinate time orders events within formal descriptions and permits
reversible equations of motion. Cost time, by contrast, marks where
reversibility fails. A system may be described as evolving forward and
backward in coordinate time within an equation, while the formation process
it represents remains irreversible due to eliminated distinctions.

This distinction clarifies a long-standing tension in physics: how
time-reversal-invariant laws coexist with irreversible phenomena. The laws
govern admissible transformations; irreversibility arises when those
transformations are instantiated through formation under observational
constraints.

\subsection{Formation Cost and the Non-Recoverability of History}
% Subsection 4.3
To say that a process "takes time" is commonly understood as a statement
about duration. Within the present framework, it is more precisely
understood as a statement about irreversible expenditure. Formation consumes
generative distinction. Once consumed, it cannot be restored without access
to information that no longer exists.

This interpretation aligns with intuitive distinctions between reversible and
irreversible processes. A reversible process can, in principle, be undone
without loss. An irreversible process cannot, because the distinctions
required for reversal have already been eliminated. Time marks this
elimination, not motion along an independent axis.

Accordingly, attempts to assign a universal quantitative measure to
formation cost misunderstand its role. Formation cost is not a quantity to
be calculated, but a boundary condition: it indicates where explanation must
give way to structural limitation. Any attempt to reconstruct eliminated
distinctions from within the observational interface conflicts with the
premise of irreversible compression.

\subsection{Prediction, Retrodiction, and Asymmetry}
% Subsection 4.4
The cost-based interpretation of time clarifies the asymmetry between
prediction and retrodiction. Prediction proceeds by applying rules to
current information to determine admissible future states. Retrodiction
seeks to reconstruct past histories from present information.

Within this framework, prediction is often tractable because it does not
require recovering eliminated distinctions. Retrodiction, by contrast,
demands access to generative paths that formation and observation have
already erased. The asymmetry is therefore structural, not merely
probabilistic or epistemic.

This perspective dissolves the expectation that a complete theory should
permit full retrodiction. A theory may be complete with respect to
admissible structures while remaining silent about realized histories.
Silence, in this context, reflects structural absence rather than
explanatory failure.

\subsection{Time, Stability, and the Appearance of Design}
% Subsection 4.5
Stable structures persist because they are robust under interaction and
perturbation. Their persistence often gives rise to an appearance of
optimization or design. When time is misconstrued as a reversible axis,
this appearance invites narratives of search, selection, or purposive
arrangement unfolding over time.

The present framework rejects such narratives. Stability does not emerge
through temporal optimization, but through survivability under formation.
Time, understood as cost, records the loss of alternatives rather than
their evaluation. The appearance of design reflects survivorship bias under
irreversible formation and observation, not intentional progression toward
preferred outcomes.

\subsection{Boundary of Temporal Explanation}
The interpretation of time advanced here imposes a clear boundary on
explanatory ambition. Questions that demand recovery of unique histories,
ultimate beginnings, or generative sequences beyond what formation preserves
do not admint answered within the observational interface. Such questions do
not fail due to ignorance; they fail because the distinctions they require
have been eliminated.

Recognizing this boundary does not diminish the role of time in physics. It
clarifies it. Time remains indispensable as a modeling parameter and as an
indicator of irreversibility, but it cannot serve as a neutral coordinate
for reconstructing formation.

Explanation ends where formation cost has been paid.
Beyond that point, no further distinction remains to be recovered.

\section{Possibility Without Proliferation: Unrealized Paths and Structural Constraint}
% Section 5
The interpretation of time as irreversible cost developed in the preceding
section provides a natural framework for reexamining the status of
unrealized possibilities. Discussions of irreversibility, indeterminacy, and
formation are frequently accompanied by appeals to parallel worlds, branching
universe, or coexisting realizations of alternative histories. Such appeals
are often motivated by a desire to preserve determinism, completeness, or
explanatory symmetry.

The present framework does not require such proliferation.
\citep{Carr2007}

The central question is not whether multiple possibilities exist, but where they exist.

\subsection{Possibility as Structural Admissibility}
% Subsection 5.1
Physical laws define a space of admissible configurations. This space may
be vast, high-dimensional, and richly constrained. Importantly, admissibility
does not entail realization. A configuration may be compatible with
governing rules without ever becoming physically instantiated.

Within the present framework, possibilities reside at the level of
structural admissibility, not at the level of realized history. They define
what could occur under given constraints, not what did occur. Possibility,
in this sense, is a feature of rule space rather than a catalog of physical
entities.

This distinction dissolves a common confusion. The existence of many
admissible paths does not imply that those paths are physically present,
coexisting, or unfolding in parallel. It implies only that formation did
not preclude them in advance.

\subsection{Formation as Commitment, Not Selection}
% Subsection 5.2
Formation converts admissible possibility into realized structure by
incurring irreversible cost. Once formation occurs, a commitment is made:
one path becomes physically instantiated, and alternative paths remain
unrealized.

Crucially, this commitment does not result from evaluation, comparison, or
selection among alternatives. Formation is not a decision procedure. It is
an irreversible physical process in which certain configurations persist
under interaction while others do not arise. Unrealized paths are not
rejected candidates; they are trajectories that never entered realization.

Accordingly, the language of "branching" should be treated with care.
Branching occurs in possibility space, not in physical history. Physical
history contains only those paths that incurred formation cost. All others
remain counterfactual.

\subsection{Parallelism Without Coexistence}
% Subsection 5.2
Many-worlds and branching-universe interpretations often rely on a picture
in which time functions as a coordinate along which all alternatives persist.
Once time is reinterpreted as cost rather than axis, this picture loses its
footing.

Parallel possibilities do not coexist as physical structures. They coexist
only as elements of admissible description. What appears as multiplicity
reflects the richness of rule space, not the multiplication of worlds.

This reinterpretation preserves the formal utility of frameworks that
employ superposition, branching, or multiplicity at the level of
representation, while declining to reify those representations as coexisting
physical realities. Representational multiplicity does not entail ontological
proliferation.

\subsection{Survivorship and the Appearance of Necessity}
% Subsection 5.4
The fact that a single realized history is observed can give rise to the
illusion of necessity: the sense that what occurred was somehow inevitable
or uniquely favored. This illusion arises from survivorship under
irreversible formation.

Because unrealized paths leave no observational residue, the realized path
appears singular and complete. The absence of alternatives at the level of
observation should not be mistaken for their absence at the level of
admissibility.

Conversely, the appearance of fine-tuning or design often results from this
same asymmetry. Only configurations capable of supporting stable structures
persist long enough to be observed. Their persistence reflects survivability,
not preference.

\subsection{Constraint Without Metaphysical Excess}
% Subsection 5.5
By locating unrealized possibilities in rule space rather than physical
coexistence, the present framework avoids metaphysical inflation without
sacrificing explanatory clarity. No additional entities are required beyond
those already implicit in admissible description and irreversible formation.

All explanatory work is done by structural constraint and survivability.
Possibility remains essential for understanding what could occur, while
realization remains essential for understanding what did occur. The two
should not be conflated.

\subsection{Summary}
% Subsection 5.6
Unrealized possibilities exist as elements of admissible structure, not as
parallel physical histories. Formation commits to a single realized path by
incurring irreversible cost, leaving alternative paths unrealized without
requiring their physical coexistence.

Multiplicity resides in description, not in reality. What persists does so because it is stable, not because it was selected from among competing worlds.

\section{Stability Constraints}
% Section 6
The following sections describe stability constraints—conditions under which
coherent operation across cognitive, decisional, observational, and runtime
layers can be maintained. These constraints are descriptive rather than
prescriptive, and do not constitute prohibitions on exploration.
As the preceding sections establish constraints on cognition, decision, and
observation across layers, it becomes necessary to address a limiting
condition that emerges at the boundary of cognitively stable operation.
This section introduces cognitive overload as a structural phenomenon arising
when the accumulation of cognitive structures, mappings, and unresolved
residuals exceeds the system's capacity for stable integration.

Cognitive overload is not treated as a failure of reasoning or an exceptional
anomaly. Rather, it represents a predictable regime encountered by systems
operating near the upper limits of cognitive and observational coherence. The
purpose of this section is not to resolve overload, but to situate it within
the framework as an identifiable boundary condition and to clarify its
implications for stability-preserving operation.


\subsection{Layered Operations and Scope Separation}
% Subsection 6.1
In this framework, system operations are strictly layer-dependent.
Cognition, decision, observation, and runtime persistence constitute
distinct layers with non-interchangeable roles. Each layer admits a specific
class of operations, and stability is preserved only when operations are
invoked within their appropriate scope.

The present analysis does not assume that higher layers subsume lower ones, nor that lower layers can be freely accessed by higher-level operations. Instead, layers are treated as operationally distinct regimes, each governed by its own constraints.

\subsection{Cognitively Stable Regions}
% Subsection 6.2
Cognition does not generate answers or actions.
Rather, it delineates a cognitively stable region under systemic
constraints, specifying the set of internal states that can be coherently
held, transformed, and related without inducing internal instability.

This cognitively stable region functions as the effective domain within which
decision mappings may be defined. States lying outside this region are not
prohibited, but they cannot be coherently operated upon by the system without
loss of stability.

\subsection{Decision as a Mapping Function}
% Subsection 6.3
A decision is not an observable event or outcome, but a mapping function
defined over the cognitively stable region.
It maps internally formed states to externally manifested transitions,
producing effects that may subsequently be rendered observable.

What is commonly identified as a "decision" corresponds only to the output of
this mapping, not to the mapping function itself. The decision function does
not reside at the observational layer and cannot be directly observed; only
its externalized consequences are accessible to observation.

\subsection{Observation as a Layer-Bound Operation}
% Subsection 6.4
Observation is an operation exclusive to the observational layer.
It presupposes a stable interface capable of producing recordable outputs and
therefore cannot be executed within the runtime layer, which consists of
ongoing, irreversible processes without separable observational interfaces.

Any attempt to perform observation at the runtime level constitutes a category
error, as it applies discrete extraction operations to processes that admit no
such decomposition. Such attempts result in systemic instability rather than
deeper access.

\subsection{Out-of-Domain Decisions and Instability Propagation}
% Subsection 6.5
When a decision mapping is invoked outside the cognitively stable region, the
resulting projection at the observational layer becomes ill-defined.
This misalignment induces instability within the individual system and,
through coupling interfaces, propagates as fluctuations in the encompassing
system.

Systemic stability therefore presupposes that decision mappings remain
confined to their cognitively defined domain. The instability observed in
larger systems is not attributed to intent or error, but to structural
misapplication of operations across layers.

\subsection{Cognitive Expansion and Observational Capacity}
% Subsection 6.6
Expansion of the cognitively stable region does not alter the layer at which
observation occurs.
However, it modulates the structure of observation by enlarging the set of
admissible mapping functions, thereby increasing the range of phenomena that
can be coherently rendered observable at the observational layer.

This expansion affects what may be observed, not where observation takes place. Layer boundaries remain invariant, while observational capacity varies with cognitive structure.
\section{Cognitive Overload and Systemic Instability}

Cognitive overload refers to regimes in which further expansion of cognitive
capacity ceases to improve coherence and instead introduces sustained
instability within the system.
In such regimes, the accumulation of internally generated models, mappings,
and unresolved residuals exceeds the system's capacity for stable
integration, producing persistent pressure on the observational layer.

Cognitive overload is not treated here as an error state or pathological
condition, but as a structural limit phenomenon intrinsic to high-capacity
systems operating near their stability boundaries. The onset of overload
indicates that additional cognitive expansion no longer yields proportional
gains in observability or decision coherence.

The present framework does not attempt to eliminate or fully resolve cognitive
overload. Rather, cognitively stable regions function to identify overload
conditions, restrict decision mappings that would further amplify instability,
and ensure sufficient damping so that systemic coherence remains bounded.

Importantly, cognitive overload does not imply a prohibition on exploration or
learning. It marks a condition under which further expansion requires
restructuring of observational capacity, rather than continued accumulation
within the existing cognitive regime.


\subsection{Boundary Statement}
% Subsection 6.7
The framework does not formalize operations that collapse layer distinctions or
attempt to exhaustively subsume runtime-level processes into observational
form.
Such operations exceed the explanatory scope of the system and introduce
irreducible instability.

The limits articulated in this section function not as prohibitions on
exploration, but as boundary conditions necessary for preserving coherence
across layered operations.

\subsection{Cognitive Overload — Summary}
Cognitive overload denotes a regime in which further cognitive expansion no
longer yields proportional gains in observability or decision coherence and
instead introduces sustained systemic pressure.
Within this framework, overload is understood as a structural limit
phenomenon, not a pathological state and not an epistemic failure.

Cognitively stable regions do not eliminate overload, nor do they provide a
means to bypass it. Their role is limited to identifying the onset of
overload conditions, constraining decision mappings that would amplify
instability, and maintaining sufficient damping to preserve coherence across
layers.

Recognizing cognitive overload as an intrinsic boundary condition allows the
framework to remain robust without claiming exhaustive control over
high-complexity regimes. In doing so, the system preserves its explanatory
integrity while explicitly acknowledging the limits of safe cognitive and
observational expansion.

\section{Conclusion: Stability, Explanation, and Structural Restraint}
% Section 7
This paper has argued that a significant class of explanatory limits in
physics arises not from incomplete knowledge, insufficient data, or
computational difficulty, but from the structural role of observation and
formation as irreversible processes. Observation functions as an interface
that compresses generative processes into reduced representations, eliminating
distinctions rather than merely concealing them. Formation incurs irreversible
cost, committing to realized structures while leaving alternative
possibilities unrealized. As a consequence, reconstruction of unique
generative histories is, in general, impossible in principle.

They function as validators of consistency rather than narrators of history.
Time, correspondingly, is not treated as a fundamental axis along which
reality is optimized or explored, but as a marker of irreversible
expenditure. What is lost ``in time'' is not position, but generative
distinction that no longer survives observational access.

Stable structures persist because they are robust under interaction and
perturbation. Their persistence can give rise to the appearance of
optimization or selection. The present analysis rejects the inference that
such appearance implies generative search, purposive arrangement, or
evaluative choice. Stability arises from survivability under formation, not
from enumeration or preference. Unrealized possibilities are not rejected
alternatives; they are paths that never incurred the irreversible cost
required for realization.

Crucially, the explanatory appeal to stability carries no normative
implication. To explain why a structure persists is not to justify its
alteration, exploitation, or domination. Stability accounts for existence; it
does not confer entitlement. This distinction is not an ethical supplement
added after the fact, but a constitutive boundary condition of the framework
itself. Any account that invokes stability as an explanatory principle while
deriving authorization, legitimacy, or license for intervention from
persistence alone misrepresents the scope of the analysis.

Human observers and agents occupy no privileged position with respect to this
boundary. Humanity is itself a variable within a stable physical world, not its
culminion or justification. The capacity to alter physical, ecological, or
informational structures does not exempt agents from the conditions of
stability under which they arose. On the contrary, increased capacity amplifies
exposure to systemic reaction. Where intervention exceeds the tolerance of
surrounding structure, instability emerges---not as moral judgment, but as
structural consequence.

Restraint, in this sense, is not a matter of ethical preference but of
survivability. It is the form taken by persistence when agency reaches a scale
capable of destabilization. The absence of restraint is not freedom; it is
transition into instability.

It is therefore necessary to clarify how restraint can emerge within human
systems at all. The present framework does not attribute restraint to moral
deliberation or normative choice, but to structural features intrinsic to
human agency itself. Characteristics commonly grouped under the heading of
``human nature''---including affect, ambiguity, bounded rationality, and
value plurality---play a stabilizing role in high-impact systems.

Pure optimization tends toward single-objective maximization, which is
generically destabilizing once agency reaches a scale capable of structural
intervention. Human non-optimality introduces friction, delay, and irreducible
plural weighting, preventing rapid convergence toward brittle extremes. In this
sense, human variability functions as a buffer rather than a defect. It limits
the speed and coherence with which intervention can be executed, thereby
preserving survivability at the system level.

Accordingly, the present framework does not support the transformation of human
systems into maximally rationalized agents. Excessive rationalization
constitutes its own form of destabilization. Stability does not require the
suppression of human characteristics; it requires that their interaction remain
within survivable bounds.

The contribution of this paper is therefore deliberately limited and
deliberately constrained. It does not propose new physical laws, revise
established formalisms, or advance metaphysical claims about ultimate origins.
Its aim is to clarify the limits of explanation imposed by irreversible
formation and observation, to distinguish descriptive stability from normative
authorization, and to delineate where explanatory demands cease to increase
structural distinction.

Where explanation ends, it ends not in mystery, but in form. What persists
does so because it is stable---not because it is permitted. Stability explains
persistence; it does not confer entitlement. Rules are identified, not
imposed. They describe stability after the fact; they do not legislate
formation.






\bigskip
\noindent
\textbf{Author's Note.} This paper is a descriptive record of structural
constraints, not a prescriptive claim to authority. The analysis is founded
upon three systemic rules that emerge inevitably from the requirement of
stability:

\begin{itemize}
\item \textbf{The Constraint on Human Agency:} Stability within the physical
and informational interface requires that the human observer remains a
variable—neither fully rationalized (AI-fied) nor idealized (deified). This
"humanity" is a necessary, non-optimized residue that prevents the total
collapse of the observational interface.
\item \textbf{Decentralized Rule-Formation:} The laws identified herein are
not decreed by the observer but are structural properties inherent to any
stable existence. The author functions merely as a decentralized node
recording these inevitabilities, not as a privileged interpreter or arbiter
of truth.
\item \textbf{Stability Without Authorization:} That a structure (such as
humanity or Earth) is a stable outcome of physical forces does not confer a
normative license for domination or unchecked intervention. Stability
accounts for persistence, not entitlement. Any agency that misinterprets its
own stability as an exemption from structural limits risks immediate systemic
destabilization—not as a moral judgment, but as a direct physical
consequence.
\end{itemize}

Where explanation ends, it ends in form. What persists does so because it is stable, not because it is permitted.
\bibliographystyle{plainnat}
\bibliography{references}

\end{document}
