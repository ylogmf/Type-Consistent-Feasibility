\documentclass[11pt]{article}

\usepackage{amsmath,amssymb,amsthm}

\title{An Observation on Set-Valued Solution Semantics in NP}
\author{Yanlin Li \\ Independent Researcher}
\date{}

\begin{document}
\maketitle

\begin{abstract}
We observe that the defining semantic content of NP is not the existence of a single solution,
but the existence of a \emph{verifiable set of witnesses}. Under this view, point solutions such as
those arising in cryptographic settings correspond to degenerate cases of set-valued solution
semantics. This unifies point, interval, and general solution-set interpretations within a single
framework. We briefly note that this perspective naturally suggests a geometric interpretation,
without developing it here.
\end{abstract}

\section{Observation}

\paragraph{NP as set-valued semantics.}
By definition, a language $L \subseteq \Sigma^*$ belongs to $\mathbf{NP}$ if there exists a
polynomial-time verifier $V$ such that
\[
x \in L \iff \exists w \; V(x,w) = 1 .
\]
For each input $x$, this induces a \emph{solution set}
\[
W(x) := \{ w \mid V(x,w) = 1 \}.
\]
Thus, the semantic content of NP is not tied to any particular witness, but to the
\emph{existence of a non-empty, efficiently verifiable set of witnesses}. Verification operates
over sets, not points.

\paragraph{Empty solution sets.}
The same set-based semantics also naturally accounts for unsatisfiable instances.
When no witness satisfies the verification predicate, we have
\[
W(x) = \varnothing .
\]
Thus, the absence of solutions is represented as the empty witness set, rather than
as a distinct semantic category.

\paragraph{Point solutions as degenerate sets.}
A ``point solution'' corresponds to the case
\[
|W(x)| = 1 ,
\]
while instances admitting multiple witnesses satisfy $|W(x)| > 1$.
All cases are uniformly represented within the same set-valued framework.


\paragraph{Unified view.}
From this perspective, point solutions, intervals of solutions, and more general regions or
collections of solutions are treated uniformly as witness sets.
The distinction lies only in the cardinality and structure of $W(x)$, not in the underlying
verification principle.
This unification highlights that NP is fundamentally a theory of verifiable solution sets,
with point solutions appearing as boundary cases.

\paragraph{Remark.}
This set-based viewpoint naturally suggests a geometric interpretation of solution structure.
We do not pursue that interpretation here.

\end{document}
