\documentclass[11pt]{article}

\usepackage{amsmath, amssymb, amsthm}
\usepackage{geometry}
\usepackage{hyperref}
\usepackage{tabularx}

\geometry{margin=1in}

\title{Observations on Time, Information Compression, and Observer Interfaces}
\author{Anonymous Observer}
\date{\today}

\newtheorem{observation}{Observation}
\newtheorem{definition}{Definition}
\newtheorem{remark}{Remark}
\newtheorem{theorem}{Theorem}
\newtheorem{conjecture}[theorem]{Conjecture}

\begin{document}

\maketitle

\begin{abstract}
This note records a sequence of observations concerning time, causality, and
quantum phenomena, framed through the concept of \emph{observer interfaces}.
Rather than proposing a finalized physical theory, we treat time as an emergent
structural effect arising from information compression, decoherence, and
observer-limited interaction. Classical continuity, quantum discontinuity,
and apparent violations of causality are examined under a unified interface-based
perspective.
\end{abstract}

\section{Motivation}

Several foundational puzzles motivate this exploration:

\begin{itemize}
  \item Why does the speed of light appear as a universal limit?
  \item Why do quantum processes appear to exhibit ``superluminal'' correlations
        without enabling communication?
  \item Why does the present moment seem to possess a finite ``thickness''?
  \item Why is the classical world continuous, while the quantum world appears
        discrete or jump-like?
  \item What role does observation play in the generation of time?
\end{itemize}

We treat these not as isolated questions, but as different projections of a
single structural phenomenon.

\section{Time as an Emergent Effect}

\begin{observation}[Time as Information Compression]
Time may be interpreted not as a primitive dimension, but as a byproduct of
information compression performed by an observer-limited interface.
\end{observation}

Under this view, temporal ordering emerges when a system compresses a large
state space into a sequence of admissible, recordable states. What is experienced
as ``flow of time'' corresponds to the rate at which distinguishable information
can be stably externalized.

\begin{remark}
This perspective aligns with the intuition that entropy increase and irreversibility
are observer-relative rather than absolute.
\end{remark}

\section{Speed of Light as an Interface Constraint}

\begin{observation}[Light-Speed as Interface Bandwidth]
The speed of light represents a maximal bandwidth constraint of spacetime
interfaces, rather than a fundamental limit on all physical processes.
\end{observation}

Information cannot be transmitted faster than this limit because doing so would
require bypassing the observer's causal interface. Quantum correlations may exceed
this limit internally, but cannot be externalized into communicable signals.

\begin{observation}[Interface-Induced Gravity]
Let $\Phi : \Omega \to \mathcal{I}$ be a non-invertible interface map from a space
of generative states $\Omega$ to an observer-accessible space $\mathcal{I}$,
induced by a fixed rule system $\Theta$.

Assume that:
\begin{itemize}
  \item $\Omega$ contains multiple realizations corresponding to identical
        interface data $i \in \mathcal{I}$;
  \item system evolution preserves admissibility under $\Theta$;
  \item observers are restricted to dynamics expressible solely in terms of
        $\mathcal{I}$.
\end{itemize}

Then the induced effective dynamics on $\mathcal{I}$ necessarily include
non-recoverable aggregation tendencies, corresponding to motion toward
equivalence classes of lower informational resolution.

In physical spacetime realizations, this aggregation tendency manifests as a
universal attractive interaction among energy-localized configurations.
We interpret gravitational attraction as the macroscopic signature of interface
non-invertibility under information-compressing observation.
\end{observation}

\subsection{Interface Extremes: Gravity and Black Holes}

\begin{observation}[Interface-Induced Gravity and Interface Collapse]
Let $\Phi : \Omega \to \mathcal{I}$ be a non-invertible interface map induced by
a fixed rule system $\Theta$, and suppose that observer-accessible dynamics are
restricted to $\mathcal{I}$.

Non-invertibility of $\Phi$ induces effective aggregation tendencies on
$\mathcal{I}$, corresponding to motion toward equivalence classes of reduced
informational resolution. At macroscopic scales, this manifests as a universal
attractive interaction among localized energy configurations, interpreted as
gravitational attraction.

A black hole corresponds to an extreme regime of this same mechanism, in which
$\Phi$ collapses large regions of $\Omega$ onto minimal interface representations,
rendering further externalization of internal state information impossible.
\end{observation}

In this unified view, gravity and black holes are not distinct physical
phenomena, but different regimes of interface non-invertibility. Gravity reflects
a \emph{gradient of informational compression}, while a black hole represents a
limit point at which interface-mediated time and state differentiation cease
to function.

\begin{remark}[Consistency with the Equivalence Principle]
This interpretation does not violate the equivalence principle.
Locally, an observer embedded within an interface-defined equivalence class
cannot distinguish between aggregation induced by interface compression and
uniform acceleration.

The equivalence principle thus arises naturally from observer confinement to
$\mathcal{I}$: locally accessible dynamics remain invariant under transformations
that preserve interface-level admissibility. Apparent gravitational effects
reflect global properties of $\Phi$, while local experiments remain insensitive
to the underlying informational gradient.
\end{remark}

\subsection{A Minimal Discrete Toy Model}

Let $\Omega = \{ \omega_1, \omega_2, \dots, \omega_N \}$ be a finite set of
generative states. Let the interface space be
$\mathcal{I} = \{ i_1, i_2, \dots, i_M \}$ with $M < N$.

Define a non-invertible interface map
\[
\Phi : \Omega \to \mathcal{I}
\]
such that multiple $\omega \in \Omega$ map to the same $i \in \mathcal{I}$.

Let $\Theta$ be a rule system enforcing admissible transitions
$\omega \to \omega'$ that preserve $\Phi$-consistency.
Observers have access only to transitions expressible on $\mathcal{I}$.

For each $i \in \mathcal{I}$, define an equivalence class
\[
[\omega]_i = \{ \omega \in \Omega \mid \Phi(\omega) = i \}.
\]

Suppose transition probabilities on $\Omega$ are uniform among admissible states.
Then interface-level dynamics induce an effective bias toward equivalence classes
with larger cardinality $|[\omega]_i|$.

\begin{observation}[Emergent Attraction from State Degeneracy]
Given observer restriction to $\mathcal{I}$, any region corresponding to a larger
preimage under $\Phi$ induces an effective attractive tendency in interface-level
dynamics.

This attraction is not a fundamental force, but a statistical consequence of
state degeneracy under non-invertible observation.
\end{observation}

In this model, a black-hole-like regime corresponds to an interface state
$i^\ast \in \mathcal{I}$ such that
\[
|[\omega]_{i^\ast}| \gg |[\omega]_i| \quad \forall i \neq i^\ast,
\]
and transitions out of $[\omega]_{i^\ast}$ are disallowed by $\Theta$.

From the interface perspective, dynamics appear frozen, while internal
micro-dynamics in $\Omega$ may continue unrestricted.

\subsection{Energy as Interface Occupancy Density}

\begin{definition}[Interface Occupancy]
Given a non-invertible interface map $\Phi : \Omega \to \mathcal{I}$, define the
\emph{interface occupancy density} of an interface state $i \in \mathcal{I}$ as
\[
\rho(i) := | \Phi^{-1}(i) |,
\]
the cardinality of its preimage in the generative space $\Omega$.
\end{definition}

\begin{observation}[Energy as Occupancy Density]
Energy may be interpreted as a measure of interface occupancy density.
Regions of high energy correspond to interface states with large degeneracy in
$\Omega$, i.e., high $\rho(i)$.
\end{observation}

Under observer-restricted dynamics on $\mathcal{I}$, transitions preferentially
enter and remain within regions of higher occupancy density. From the interface
perspective, such regions behave as persistent, inertia-bearing configurations.

\begin{observation}[Inertia from Occupancy Persistence]
Interface states with high occupancy density exhibit resistance to change in
interface-level dynamics, as transitions out of large equivalence classes are
statistically suppressed.

This persistence manifests as inertial behavior.
\end{observation}

\begin{remark}[Inevitability of Large-Scale Structure Formation]
At cosmological scales, any non-invertible interface map $\Phi$ combined with
observer-restricted dynamics on $\mathcal{I}$ necessarily induces uneven
occupancy distributions.

Even from initially homogeneous conditions in $\Omega$, stochastic fluctuations
produce interface regions with slightly higher occupancy density. These regions
subsequently attract further interface-level trajectories, amplifying density
contrast over time.

Thus, large-scale structure formation is not contingent on fine-tuned initial
conditions, but is a generic consequence of interface non-invertibility and
information compression.
\end{remark}

\section{Interface Compression and Emergent Attraction}

\subsection{Setup}

Let $\Omega$ be a space of microstates (generative states). Let $\mathcal{I}$ be a space of
observer-accessible interface data (records, observations, transcripts, measurement outcomes).

\begin{definition}[Interface map]
An \emph{interface map} is a (possibly stochastic) map
\[
\Phi:\Omega \to \mathcal{I},
\]
induced by a fixed rule system $\Theta$ specifying what can be observed, queried, or verified.
\end{definition}

For $i\in\mathcal{I}$ define the fiber (preimage)
\[
\Omega_i := \Phi^{-1}(i)\subseteq \Omega .
\]

\begin{definition}[Interface indistinguishability]
Two microstates $\omega,\omega'\in\Omega$ are \emph{interface-equivalent} if $\Phi(\omega)=\Phi(\omega')$.
\end{definition}

\subsection{Information under an interface}

Assume a prior measure $\mu$ on $\Omega$ (or a family $\mu_\lambda$ indexed by scale $\lambda$).
This induces a distribution on $\mathcal{I}$:
\[
P(i) := \mu(\Omega_i).
\]

\begin{definition}[Interface surprisal / compressed description length]
Define the interface description length
\[
\ell(i) := -\log P(i),
\]
and the interface entropy
\[
H(\Phi) := \mathbb{E}_{i\sim P}[\ell(i)] = -\sum_{i\in\mathcal{I}} P(i)\log P(i).
\]
\end{definition}

Intuition: $\ell(i)$ measures how many bits are required to specify the observed interface datum $i$,
given the rule-limited access encoded by $\Phi$.

\subsection{Coarse-grained geometry and effective dynamics}

Let $x\in\mathbb{R}^3$ denote a coarse-grained spatial descriptor extracted from $i$,
written $x = \pi(i)$ for some projection $\pi:\mathcal{I}\to\mathbb{R}^3$ (e.g., center-of-mass record).

Define an \emph{effective free-energy functional} on interface data:
\[
\mathcal{F}(i) := \mathbb{E}[E \mid i] - T\,\ell(i),
\]
where $E$ is the microstate energy and $T$ is an effective temperature/scale parameter
(used only as a Lagrange multiplier controlling information contribution).

We posit an induced drift (gradient flow) on accessible coordinates:
\[
\dot{x} \;\propto\; -\nabla_x\, \mathbb{E}[\mathcal{F}(i)\mid \pi(i)=x ].
\]
In regimes where the $\ell(i)$ term dominates the spatial dependence,
the drift is toward configurations minimizing description length (maximizing compressibility).

\subsection{Observation (Interface Compression $\Rightarrow$ Emergent Attraction)}
\begin{observation}[Gravity as an interface-level effect]
If the interface map $\Phi$ is non-invertible and the induced description length $\ell(i)$
strictly decreases under aggregation of energy-localized configurations (in the sense that
merging two separated lumps yields lower $\ell(i)$ at fixed total energy),
then the induced effective dynamics on coarse-grained coordinates exhibit a systematic
attractive drift between those lumps.
\end{observation}

\subsection{Conjectures}

\begin{conjecture}[Generic non-invertibility of physical interfaces]
For physically realizable rule systems $\Theta$ with bounded observation bandwidth,
the induced interface map $\Phi:\Omega\to\mathcal{I}$ is generically non-invertible:
\[
\exists\,\omega\neq\omega' \in \Omega \;\; \text{s.t.}\;\; \Phi(\omega)=\Phi(\omega').
\]
\end{conjecture}

\begin{conjecture}[Compression monotonicity under aggregation]
In spatially extended closed systems with conserved energy--momentum, the interface description length
$\ell(i)$ is generically \emph{aggregation-subadditive}:
\[
\ell(i_{\text{together}}) \;\le\; \ell(i_{\text{separate}}) - c(d),
\]
where $d$ is separation distance and $c(d)>0$ increases with $d$ in the regime where
the interface cannot resolve full microstructure.
\end{conjecture}

\begin{conjecture}[Entropic-force form of attraction]
In the aggregation-subadditive regime, the induced drift admits an entropic-force form:
\[
F_{\text{eff}}(d) \;=\; -\frac{\partial}{\partial d}\Big(T\,\mathbb{E}[\ell(i)\mid d]\Big),
\]
which is attractive whenever $\mathbb{E}[\ell(i)\mid d]$ increases with separation $d$.
\end{conjecture}

\begin{conjecture}[Newtonian scaling as a universal low-resolution limit]
Assume (i) translation/rotation invariance at the interface level, (ii) locality of interface resolution,
and (iii) far-field dominance of a single scalar ``compressibility potential''.
Then, in the long-distance limit, the leading attractive term satisfies
\[
F_{\text{eff}}(d) \;\approx\; -\,G_{\text{eff}}\frac{m_1 m_2}{d^{2}},
\]
for some interface-dependent constant $G_{\text{eff}}$ determined by $\Theta$ and the coarse-graining scale.
\end{conjecture}

\begin{conjecture}[Equivalence principle as interface universality]
If $\Phi$ erases internal labels beyond total energy--momentum, then the induced acceleration
depends only on the interface-relevant energy localization and not on micro-composition:
\[
a_{\text{eff}}(x) \;\text{is independent of microstate species given fixed coarse energy profile.}
\]
\end{conjecture}

\begin{conjecture}[Rule-external augmentation changes ``gravity'']
Augmenting the rule system $\Theta$ with rule-external structure (higher-resolution access)
refines $\Phi$ and reduces fiber sizes $|\Omega_i|$, thereby weakening the compression term
and altering the effective attraction:
\[
\Theta \subset \Theta' \;\Rightarrow\; \ell_{\Theta'}(i) \le \ell_{\Theta}(i)
\quad\text{and}\quad
|F_{\text{eff}}^{\Theta'}| \le |F_{\text{eff}}^{\Theta}|
\;\text{in comparable regimes.}
\]
\end{conjecture}

\section{Quantum Superposition and Apparent Atemporality}

\begin{definition}[Pre-Temporal State]
A system is in a \emph{pre-temporal state} if its internal evolution is not yet
projected onto an observer-defined temporal ordering.
\end{definition}

\begin{observation}[Quantum Superposition as Pre-Temporal]
Quantum superposition may be understood as a state that has not yet entered
observer-generated time.
\end{observation}

Measurement is not merely a passive readout, but an active interface event that
projects the system into a temporal sequence.

\section{Measurement as a Local Time-Generation Event}

\begin{observation}[Measurement Generates Local Time]
Each quantum measurement event constitutes a localized generation of time,
establishing a before-and-after relation relative to the observer.
\end{observation}

This explains why time appears continuous at macroscopic scales (due to dense,
overlapping measurement events), yet discontinuous at microscopic scales.

\section{The Thickness of ``Now''}

\begin{observation}[Finite Present Interval]
The ``present'' is not a mathematical instant, but a finite interval required
for information stabilization and interface registration.
\end{observation}

The thickness of ``now'' corresponds to the minimum temporal window necessary
for coherent observation.

\section{Classical Continuity vs Quantum Discreteness}

\begin{observation}[Continuity as High-Frequency Averaging]
Classical continuity emerges from the averaging of an enormous number of
microscopic, discrete interface events.
\end{observation}

\begin{observation}[Interface-Induced Electric Interaction]
If an interface map $\Phi:\Omega\to\mathcal{I}$ selectively preserves certain
signed or oriented degrees of freedom while compressing others, then asymmetries
in interface occupancy may persist under aggregation.

Such persistent interface-level asymmetries induce effective long-range
interactions that are not universally attractive. In physical realizations,
these interactions manifest as electric charge and electromagnetic forces.

In contrast to gravity, which arises from scalar compression of interface
occupancy, electric interactions arise from vector- or sign-sensitive interface
constraints, allowing both attraction and repulsion.
\end{observation}

Quantum jumps are visible only when the observer interface has insufficient
resolution or insufficient averaging depth.

\section{Black Holes as Interface Collapse}

\begin{observation}[Black Holes as Temporal Compression Extremes]
A black hole may be viewed as a region where time is maximally compressed,
and observer interfaces fail to externalize internal states.
\end{observation}

From the external perspective, information appears frozen; from the internal
perspective, standard temporal notions may cease to apply.

\section{Free Will and Interface Choice}

\begin{observation}[Free Will as Path Selection Freedom]
Free will may be interpreted as residual freedom in selecting among admissible
compression paths within an observer interface.
\end{observation}

This freedom does not violate physical law, but operates within underdetermined
regions of the interface.

\section{The Universe Before Time}

\begin{observation}[Atemporal Universe Hypothesis]
Prior to global decoherence, the universe may be regarded as effectively
atemporal, with no observer-relative time ordering.
\end{observation}

Under this view, the ``origin'' of the universe is not a temporal event, but a
transition into a regime where observation and time become meaningful.

\section{Stronger Observers}

\begin{observation}[Observer Hierarchy]
An observer with a more expressive interface would perceive a richer causal
structure, potentially accessing relations that appear acausal or paradoxical
to weaker observers.
\end{observation}

Causality itself may therefore be observer-relative.

\begin{table}[h]
\centering
\renewcommand{\arraystretch}{1.25}
\begin{tabularx}{\linewidth}{|l|X|X|}
\hline
\textbf{Phenomenon} 
& \textbf{Structural Cause (Interface Perspective)} 
& \textbf{Effect of Interface Enhancement} \\
\hline
Gravity 
& Information compression across spatial degrees of freedom due to interface non-invertibility 
& Effective attraction weakens; trajectories become less biased as finer spatial distinctions are resolved \\
\hline
Energy 
& Interface occupancy density (cardinality of microstate equivalence classes) 
& Effective mass or inertia decreases as degeneracy is resolved into distinguishable microstates \\
\hline
Speed of light limit 
& Finite causal bandwidth of the observer interface 
& Maximum signal speed increases only if interface bandwidth is fundamentally expanded \\
\hline
Arrow of time 
& Irreversible information compression under observation 
& Temporal asymmetry diminishes as reversible micro-dynamics become externally accessible \\
\hline
Quantum nonlocal correlations 
& Projection of generative-level nonlocality through a constrained interface 
& Correlations appear less paradoxical as intermediate generative structure becomes observable \\
\hline
No superluminal communication 
& Inability to control or steer interface fibers 
& Limited signaling may emerge only if the interface allows directed control of microstate selection \\
\hline
\end{tabularx}
\caption{Physical phenomena interpreted as consequences of interface structure, together with their expected modification under enhanced interface resolution or access.}
\end{table}

\begin{table}[ht]
\centering
\renewcommand{\arraystretch}{1.25}
\begin{tabularx}{\linewidth}{|l|X|X|X|}
\hline
\textbf{Interaction} 
& \textbf{Interface-Retained Structure} 
& \textbf{Compression Type} 
& \textbf{Characteristic Interface Behavior} \\
\hline
Gravitational 
& Scalar occupancy density (total degeneracy) 
& Fully scalar, sign-erasing compression 
& Universally attractive; unscreenable; acts on all energy-localized configurations \\
\hline
Electromagnetic 
& Signed or vector-valued interface attributes (charge, orientation) 
& Partial compression preserving sign and direction 
& Attractive or repulsive; screenable; long-range due to retained vector structure \\
\hline
Weak 
& Chiral and flavor-asymmetric interface labels 
& Selective compression violating parity symmetry 
& Short-range; non-universal; induces irreversible transmutations at the interface level \\
\hline
Strong 
& Color-like multi-component relational constraints 
& Minimal compression of internal relational structure 
& Confining; saturating; prevents isolation of interface-incomplete states \\
\hline
\end{tabularx}
\caption{The four fundamental interactions interpreted as consequences of which structural features are retained or erased by the observer interface under information compression.}
\end{table}


These phenomena are therefore not absolute laws, but stable interface-level
regularities that persist only under fixed observational constraints.

\section{Clarification: Beyond Reductionism and Informationism}

The interface-based perspective adopted in this work should not be interpreted
as a form of reductionism, nor as an ontological claim that reality is
fundamentally ``made of information.''

First, this framework does not reduce physical phenomena to a single primitive
substance or microphysical mechanism. Microstates in $\Omega$ are not eliminated,
replaced, or declared irrelevant. Rather, the analysis concerns how observer-accessible
regularities arise when dynamics are necessarily expressed through a constrained
interface $\mathcal{I}$. The focus is structural, not eliminative.

Second, the present approach is not informationism. Information here is not treated
as an intrinsic substance or universal currency of reality. Instead, informational
quantities such as description length or entropy are explicitly interface-relative:
they are defined only with respect to a rule system $\Theta$ that governs what can
be observed, recorded, or externalized. No claim is made that information exists
independently of such interfaces.

In this sense, the framework is neither bottom-up reductionist nor top-down
informational. It is relational and operational: physical phenomena are understood
as emergent constraints on dynamics induced by the necessity of observation under
limited, non-invertible interfaces. Gravity, time, and causality are not reduced
to information, but arise as stable interface-level regularities that persist across
a wide range of underlying microphysical realizations.

\begin{observation}[Existence of Stronger Interfaces]
If observable reality is mediated by a non-invertible interface that induces
irreversible information compression, and if interface fidelity admits a
partial ordering, then the existence of interfaces with strictly greater
information retention than the human-observer interface cannot be excluded
and is generically expected.

Such interfaces need not correspond to entities within the observable domain,
as any interaction with them would necessarily be projected through the weaker
interface and thereby lose its distinguishing structure.
\end{observation}

\section{Conclusion}

These observations suggest that time, causality, and physical law are not
absolute primitives, but emergent structures arising from observer interfaces,
information compression, and decoherence. This framework does not negate existing
physical theories, but offers an interpretive layer that may unify classical,
quantum, and cosmological phenomena.

\end{document}

\section*{Citation}

Li, Yanlin.
\emph{<TITLE>}.
Version~v0.4, 2025.
Available at: \url{https://github.com/ogmf/Type-Consistent-Feasibility/tree/v0.4}.

\noindent Please cite the version you consulted.

\bibliographystyle{plain}
\bibliography{references}
