\documentclass[11pt]{article}

\usepackage{amsmath,amssymb,amsthm}
\usepackage{geometry}
\geometry{margin=1in}

\newtheorem{note}{Note}
\newtheorem{proposition}{Proposition}
\newtheorem{remark}{Remark}

\title{On the Non-Existence of Universal $\mathbf{P}/\mathbf{NP}$ Classifiers}

\author{Yanlin Li}
\date{}

\begin{document}
\maketitle

\begin{abstract}
This note clarifies a structural limitation on complexity classification.
Even when a rule and parameter system is fully fixed, there is in general
no universal procedure that takes an arbitrary problem description and
decides whether it belongs to $\mathbf{P}$ or $\mathbf{NP}$.
We further observe that such classification may become possible once the
admissible rule system and problem description space are sufficiently restricted.
\end{abstract}

\section*{Setting}

Let $\Theta$ denote a fixed parameter system specifying admissible encodings,
rules, and interpretation constraints under which problems induce languages
$L(\Theta)\subseteq\Sigma^*$.
We write $\mathbf{P}(\Theta)$ and $\mathbf{NP}(\Theta)$ for feasibility predicates
evaluated relative to this fixed system.

A \emph{problem description} $q$ is assumed to induce, together with $\Theta$,
a language $L(q,\Theta)$.

\section*{No Universal Classifier in General}

\begin{note}
There does not exist, in general, a total computable function
\[
F(q,\Theta)\in\{\mathbf{P},\mathbf{NP}\}
\]
that correctly decides, for all problem descriptions $q$ and parameter systems
$\Theta$, whether $L(q,\Theta)\in\mathbf{P}$ or $L(q,\Theta)\in\mathbf{NP}$.
\end{note}

\begin{proof}[Proof sketch]
Membership in $\mathbf{P}$ or $\mathbf{NP}$ is a nontrivial semantic property of
the induced language $L(q,\Theta)$, defined via existential quantification over
algorithms (e.g., the existence of a polynomial-time decider or verifier).
Under any sufficiently expressive description formalism, problem instances
$(q,\Theta)$ can uniformly encode arbitrary Turing-machine behavior.
A universal classifier deciding $\mathbf{P}/\mathbf{NP}$ membership would therefore
decide undecidable properties as a special case.
Hence no such total computable function exists in full generality.
\end{proof}

\section*{Restricted Rule Systems}

\begin{proposition}
If, in addition to fixing $\Theta$, the admissible problem descriptions are
restricted to a decidable fragment $\mathcal{Q}_\Theta$ whose feasibility
classification is syntactically characterizable, then a total classifier
\[
F_\Theta:\mathcal{Q}_\Theta\to\{\mathbf{P},\mathbf{NP}\}
\]
may exist.
\end{proposition}

\begin{proof}[Justification]
This situation arises when $\Theta$ collapses the space of admissible problems
into a family with effective classification rules.
Typical cases include:
\begin{itemize}
  \item Finite admissible problem catalogs, where membership is decidable by lookup.
  \item Syntactically restricted fragments with known polynomial-time bounds.
  \item Rule systems admitting tractability/intractability dichotomies under fixed encodings.
\end{itemize}
In such settings, feasibility classification becomes computable by construction.
\end{proof}

\begin{remark}
Fixing $\Theta$ renders feasibility \emph{comparisons} well-typed, but does not,
by itself, guarantee the existence of a universal mechanical classifier.
Classification becomes possible only after further restricting the rule system
and admissible description space.
\end{remark}


\end{document}

\section*{Citation}

Li, Yanlin.
\emph{<TITLE>}.
Version~v0.4, 2025.
Available at: \url{https://github.com/ogmf/Type-Consistent-Feasibility/tree/v0.4}.

\noindent Please cite the version you consulted.

\bibliographystyle{plain}
\bibliography{references}
