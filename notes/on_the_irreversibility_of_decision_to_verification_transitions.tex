\documentclass[11pt]{article}
\usepackage{amsmath, amssymb, amsthm}
\usepackage[margin=1in]{geometry}

\newtheorem{definition}{Definition}
\newtheorem{theorem}{Theorem}
\newtheorem{remark}{Remark}

\title{Interface Irreversibility Under Rule Externalization}
\author{YANLIN LI}
\date{\today}

\begin{document}
\maketitle

\begin{abstract}
We formalize a decision interface and a verification interface as distinct
computational contracts. We define a rule-externalization operation that moves
rule-structure from the decision procedure into auxiliary information.
We prove an interface-irreversibility statement: once rules are externalized,
recovering a decision procedure without re-internalizing the same rule-structure
is not well-defined as an interface transformation.
\end{abstract}

\section{Motivation}
Decision procedures presuppose that the criteria governing admissibility
are fixed and explicitly available within the computational process itself.
Given an input $x$, a decision procedure is expected to determine membership
by evaluating $x$ against an internally realized rule structure, producing
an unconditional accept or reject outcome.

\noindent
When such rule structure is externalized or left implicit, the computational
task changes in character. Correctness can no longer be established by direct
evaluation on $x$ alone, but only relative to auxiliary information specifying
how admissibility should be assessed. The task thus shifts from decision to
verification: from determining membership outright to checking consistency
with externally supplied conditions.

\section{Two Interfaces}

\begin{definition}[Rule system]
A rule system is a predicate $R(x)$ over inputs $x \in \Sigma^*$ determining
admissible membership for a target language $L_R = \{x : R(x)=1\}$.
\end{definition}

\begin{definition}[Decision interface]
A decision interface for $R$ is an algorithm (or machine) $D_R$ such that,
for all $x$, $D_R(x)=R(x)$ and $D_R$ is evaluated under a fixed internal rule
realization of $R$.
\end{definition}

\begin{definition}[Verification interface]
A verification interface is a polynomial-time predicate $V(x,y)$ such that
\[
x \in L \iff \exists y \; V(x,y)=1.
\]
Here $y$ is auxiliary information (a witness, certificate, or rule-encoding).
\end{definition}

\section{Externalization as an Interface Transformation}

\begin{definition}[Rule-externalization operator]
Let $E$ be an operator that transforms a rule system $R$ into a relation
$V_R(x,y)$ where $y$ encodes missing rule structure needed to establish correctness.
We say $E$ is sound if for all $x$:
\[
R(x)=1 \iff \exists y \; V_R(x,y)=1.
\]
\end{definition}

\begin{remark}
$E$ preserves the underlying language family in an extensional sense (membership),
but changes the contract by which correctness is established: from direct decision
to relative verification.
\end{remark}

\section{Irreversibility}

\begin{definition}[Interface-preserving recovery map]
A recovery map $F$ is interface-preserving if it takes a verifier $V_R(x,y)$
and returns a decision procedure $\widehat{D}$ such that $\widehat{D}(x)=R(x)$
for all $x$, \emph{without} access to any particular witness $y$ at runtime.
\end{definition}

\begin{remark}[Observation: Interface irreversibility under rule externalization]
Decision and verification constitute distinct computational interfaces.
A decision interface evaluates membership as a total function
$x \mapsto R(x)$ under an internally fixed rule structure.
A verification interface evaluates a relation $(x,y) \mapsto V(x,y)$
relative to auxiliary information $y$.
\par\noindent
When rule structure is externalized into $y$, the verification interface
no longer determines a unique decision function on inputs $x$ alone.
Any attempt to recover a decision procedure from such a verifier
must either (i) re-internalize rule-structure equivalent to that encoded in $y$,
or (ii) leave the resulting decision behavior underdetermined.
\par\noindent
This reflects an interface-level irreversibility: the transition from
decision to verification preserves extensional membership but discards
the functional contract required for decision. The loss cannot be reversed
by interface manipulation alone.
\end{remark}

\begin{proof}[Proof sketch]
Decision is a functional contract $x \mapsto R(x)$; verification is a relational
contract $(x,y)\mapsto V_R(x,y)$ with existential quantification over $y$.
If $y$ encodes rule-structure not determined by $x$, then multiple distinct
rule-realizations can induce the same verification behavior on some pairs,
while differing on the induced function $R(x)$.

\noindent
Any $F$ that outputs $\widehat{D}(x)$ from the verifier alone must, for each $x$,
resolve which rule-realization the existential quantifier is referencing.
But resolving this requires selecting (implicitly or explicitly) the missing
rule-structure---precisely the information externalized into $y$.

\noindent
Therefore, either (i) $F$ is not well-defined as an interface transformation
(it cannot uniquely determine $\widehat{D}$ from the verification contract),
or (ii) $F$ reconstructs and embeds an equivalent rule system internally,
collapsing back to a decision interface by re-internalization.
\end{proof}

\section{Implications and Scope}
This note does not address the resolution of the classical P versus NP
question. No claims are made regarding the relative power of deterministic
and nondeterministic computation under standard formulations.

\noindent
The purpose of the discussion is limited to well-formedness considerations.
Specifically, we examine how changes in the computational interface—such as
the externalization of rule structure—alter the meaning of feasibility
predicates. The observations concern interface shifts rather than complexity
separations.


\end{document}

\section*{Citation}

Li, Yanlin.
\emph{<TITLE>}.
Version~v0.4, 2025.
Available at: \url{https://github.com/ogmf/Type-Consistent-Feasibility/tree/v0.4}.

\noindent Please cite the version you consulted.

\bibliographystyle{plain}
\bibliography{references}
